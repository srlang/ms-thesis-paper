
\subsubsection*{Forgiving Punishments}
\label{sec:findings-expts-punishments}

%%%
Since it was deemed likely that being in a losing position early in the game
tended to never recover and thus result in a loss,
it followed that punishing losing states for something beyond the agent's
control was slightly unfair.
%
Furthermore,
it was postulated that since the punishment mechanism in effect cycled
strategies
and that there was a possibility that an occasionally winning strategy was often
outweighed by the tendency to lose,
a less strict method of punishment could be used to ensure that the occasional
win from a losing position remains visible.
%
As such,
the update step for modifying the weights was adjusted slightly
so that the constant adjustment factor was significantly smaller for losing
games than it was for winning games.
%
Instead of using the adjustment constant of
$C = s \cdot (\text{PlayerScore} - \text{OppScore})$,
for both winning and losing agents,
the losing agent's adjustment factor was defined as
$C = \frac{1}{4} s \cdot (\text{PlayerScore} - \text{OppScore})$.
%%%

\paragraph*{Results}

%%%
% TODO: results
%%%
