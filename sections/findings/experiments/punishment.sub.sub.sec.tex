
\subsubsection*{Punishment Severity}
\label{sec:findings-expts-punishments}

%%%
Under the assumption that a player in a losing position
early in the game tended to never recover and thus lost as a result of
unfortunate positioning,
it followed that punishing losing states for something beyond the agent's
control was potentially unfair.
%
Furthermore,
it was postulated that since the punishment mechanism, in effect, cycled
strategies
and that there was a possibility that an occasionally winning strategy was often
outweighed by the tendency to lose,
a less strict method of punishment could be used to ensure that the occasional
win from a losing position remains visible.
%
As such,
the update step for modifying the weights was adjusted slightly
so that the constant adjustment factor was significantly smaller for losing
games than it was for winning games.
%
Instead of using the adjustment constant of
$C = s \cdot (\textit{PlayerScore} - \textit{OpponentScore})$
for both winning and losing agents,
the losing agent's adjustment factor was defined as
$C = \frac{1}{4} s \cdot (\textit{PlayerScore} - \textit{OpponentScore})$.
%
The reduction to $\frac{1}{4}$ was made arbitrarily for illustrative purposes.
%%%

\paragraph*{Results}

%%%
The introduction of an amount of forgiveness did not lead to any worthwhile
difference in learned policy.
%
In contrast to the goal of allowing an occasionally good strategy to form
in losing positions,
those positions are even less sure as to which strategy to take,
as seen in Figure~\ref{fig:findings-expts-punish-strats}.
%
Therefore,
it may be concluded that the increased likelihood of losing is 
likely not unfairly punishing potentially good recovery policies.
%
It can be argued that $\frac{1}{4}$ was not a proper ratio
to compensate for the likelihood of loss,
but the similarity of patterns learned and decreased certainty
indicate that the punishment mechanism is functioning adequately.
%
Similarly,
while reducing the amplitude of changes
does lead to a more evenly mixed losing policy,
this is effectively the result of a learning rate adjustment.
%
As will be shown in the learning rate experiments,
these forms of adjustments do not lead to differences in learned behaviors.
%%%

% fig:findings-expts-punish-strats

\begin{figure}
\center

	\begin{subfigure}[t]{0.22\textwidth}
		\includegraphics[width=\stratgraphwidth]{images/findings/experiments/punishment/strategies_handmaxmin.png}
		\caption{\handmaxmin}
	\end{subfigure}
	~
	\begin{subfigure}[t]{0.22\textwidth}
		\includegraphics[width=\stratgraphwidth]{images/findings/experiments/punishment/strategies_handmaxavg.png}
		\caption{\handmaxavg}
	\end{subfigure}
	~
	\begin{subfigure}[t]{0.22\textwidth}
		\includegraphics[width=\stratgraphwidth]{images/findings/experiments/punishment/strategies_handmaxmed.png}
		\caption{\handmaxmed}
	\end{subfigure}
	~
	\begin{subfigure}[t]{0.22\textwidth}
		\includegraphics[width=\stratgraphwidth]{images/findings/experiments/punishment/strategies_handmaxposs.png}
		\caption{\handmaxposs}
	\end{subfigure}

	\begin{subfigure}[t]{0.22\textwidth}
		\includegraphics[width=\stratgraphwidth]{images/findings/experiments/punishment/strategies_cribminavg.png}
		\caption{\cribminavg}
	\end{subfigure}
	~
	\begin{subfigure}[t]{0.22\textwidth}
		\includegraphics[width=\stratgraphwidth]{images/findings/experiments/punishment/strategies_peggingmaxavggained.png}
		\caption{\peggingmaxavggained}
	\end{subfigure}
	~
	\begin{subfigure}[t]{0.22\textwidth}
		\includegraphics[width=\stratgraphwidth]{images/findings/experiments/punishment/strategies_peggingmaxmedgained.png}
		\caption{\peggingmaxmedgained}
	\end{subfigure}
	~
	\begin{subfigure}[t]{0.22\textwidth}
		\includegraphics[width=\stratgraphwidth]{images/findings/experiments/punishment/strategies_peggingminavggiven.png}
		\caption{\peggingminavggiven}
	\end{subfigure}

\caption{
	Final strategy graphs for an agent which has less severe punishment
	after training for one million games.
}
\label{fig:findings-expts-punish-strats}
\end{figure}


