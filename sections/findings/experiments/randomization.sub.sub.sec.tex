
\subsubsection{Random Exploration}
\label{sec:findings-expts-rand}

%%%
While analysis was being performed on the results of Round 2 and further
experimentation was underway,
the discovery was made that the method for deciding which combination of cards
should be picked was not as expected
thanks to a overlooking the matter during development.
%
A ``correct'' method of choosing cards would have selected the card combination
corresponding to the maximum value found in the produced $p$ vector,
breaking ties randomly,
with a chance of making an exploration step by selecting a combination at
uniform random.
%
The implemented mechanism did indeed normally choose the maximally represented
combination in the $p$ vector,
but random exploration was performed using the $p$ vector as a probability
distribution.
%
Further complicating the issue,
a review of the code was misunderstood and it was wrongly concluded that
the $p$ vector was normally used as a probability distribution
and random exploration was carried out at uniform.
%%%

%%%
Thanks to the nature of the produced $p$ vectors,
it was assumed that the implemented solution would not make make a marked
difference in performance or learning.
%
This is because $p$ for locations which were untrained 
ended up producing $p$ vectors which were remarkably close to
uniform themselves.
%
However,
states which were trained often produced $p$ vectors in which
three or four combinations of cards were almost equally highly weighted
while unpopular combinations had nearly zero chance of being chosen.
%
While this resulting $p$ vector was created exactly as intended,
the skewing towards certain combinations means that so-called
random exploration steps are unlikely to actually explore policy options
which have no recent bias
in later, trained states.
%
This may result in only a negligible effect since
only highly trained locations will show this bias
and may even be helpful as it allows a set of weights to converge more quickly
to a stable policy.
%%%


\paragraph{Results}
\label{sec:findings-expts-rand-results}

%%%
% Results of the given experimental change(s) as they are generated.
%%%

