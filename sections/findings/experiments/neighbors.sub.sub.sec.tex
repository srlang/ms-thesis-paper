
%%%
% Discuss how neighboring weights were applied and what the results were
% TODO: results
%%%

\subsubsection{Neighboring Weights}

%%%
% Process
%%%

%%%
In order to smooth out the strategy graphs
and prevent isolated states of separate weights,
a blending of neighboring weights was developed.
%
Rather than simply take the set of weights allocated to a single score location,
the agent instead takes a weighted average of all surrounding weight vectors
with its own location.
%
In other words,
\[
    w'_{m,o,d} = %\mathrm{l1norm}\left(
    \left|
    Xw_{m,o,d} +
    Y \sum_{i\in\{-1,0,1\}} \sum_{j\in\{-1,0,1\}} w_{m+i,o+j,d}
    \right|_1
    %\right)
\]
where $X$, $Y$ are ratios of each vector's effect, $X+Y = 1$,
and $m+i$, $o+j \in [0,120]$.
%
The desired effect was to allow a score location to learn from its neighbors
so that a neighborhood effect was present in the decision.
%%%

