
\subsubsection*{Learning Rate Adjustment}
\label{sec:findings-expts-learnrate}

%%%
In order to determine if the learning rate was too high in Round 2,
even though it had been significantly reduced from Round 1,
a varying amount of learning rates were tried.
%
These runs were intended to see if an optimal policy was being overstepped by
making too large of an adjustment.
%%%


\paragraph*{Results}

%%%
% TODO: verify results when experiment has been rerun
% and choose correct set of answers
%%%

%%%
% 2 possibilities foreseeable for results:
%	1. patterns are the same
%	2. something new is learned
%%%


%%%
As can be seen in Figure~\{ref{expts-lr-comp},
the same set of behavioral patterns are learned as in previous training
sessions.\footnote{
	As with the decay section,
	this part needs to be rerun,
	so accuracy will need to be verified.
	% TODO: remove footnote
	What's more: since this section is speculative,
	only one of either this paragraph or the next one will actually be used.
}
%
Therefore,
it is safe to speculate that some optimum is not being stepped over,
allowing a worse set of weights to be pursued instead.
%
Thus,
the agent is correctly learning to solve the problem.
%
Ergo,
the problem setup itself or else the system of rewards
must be flawed in some manner.
% TODO: ^ if this is the case, consider opening the sanitycheck section with
% a pointing out of this
%
However,
a decrease in scaling factor leads to slower adjustments,
showing that the scaling factor does indeed function as a learning rate.
%%%

%%%
% TODO: ^ or v: choose one
%%%

%%%
As can be seen in Figure~\{ref{expts-lr-comp},
a different set of behavioral patterns are learned when the
scaling factor,
i.e. the learning rate,
is lowered to \${VALUE}. % TODO: <-- value
%
This serves to show that a better policy was previously being skipped over
by previous training sessions.
%
Were time to permit,
this new learning rate would be further trained to find a more optimal
policy.
%
Alas, this is no longer an option,
so the results of such an inquiry are left for speculation
or for future research.
% TODO: ^ if this is the case, probably include in discussion section
%%%

% fig:expts-lr-comp
\begin{figure}[h]
	\centering

	\begin{tabular}{c | c c c c}
		% Outline:
		%   s\g |  250k | 500k | 750k | 1mm
		%	0.25
		%   0.50
		%   1.00
		%   1.50
		$s$\textbackslash game & 250,000 & 500,000 & 750,000 & 1,000,000 \\
		\hline
		0.25 & % a & b & c & d
			\includegraphics[width=2cm]{images/findings/experiments/learning_rate/lr_025_250.png} & % 250
			\includegraphics[width=2cm]{images/findings/experiments/learning_rate/lr_025_500.png} & % 500
			\includegraphics[width=2cm]{images/findings/experiments/learning_rate/lr_025_750.png} & % 750
			\includegraphics[width=2cm]{images/findings/experiments/learning_rate/lr_025_1mm.png} \\ % 1mm
		0.50 & 
			\includegraphics[width=2cm]{images/findings/experiments/learning_rate/lr_050_250.png} & % 250
			\includegraphics[width=2cm]{images/findings/experiments/learning_rate/lr_050_500.png} & % 500
			\includegraphics[width=2cm]{images/findings/experiments/learning_rate/lr_050_750.png} & % 750
			\includegraphics[width=2cm]{images/findings/experiments/learning_rate/lr_050_1mm.png} \\ % 1mm
		1.00 & 
			\includegraphics[width=2cm]{images/findings/experiments/learning_rate/lr_100_250.png} & % 250
			\includegraphics[width=2cm]{images/findings/experiments/learning_rate/lr_100_500.png} & % 500
			\includegraphics[width=2cm]{images/findings/experiments/learning_rate/lr_100_750.png} & % 750
			\includegraphics[width=2cm]{images/findings/experiments/learning_rate/lr_100_1mm.png} \\ % 1mm
		1.50 & 
			\includegraphics[width=2cm]{images/findings/experiments/learning_rate/lr_150_250.png} & % 250
			\includegraphics[width=2cm]{images/findings/experiments/learning_rate/lr_150_500.png} & % 500
			\includegraphics[width=2cm]{images/findings/experiments/learning_rate/lr_150_750.png} & % 750
			\includegraphics[width=2cm]{images/findings/experiments/learning_rate/lr_150_1mm.png} \\ % 1mm
	\end{tabular}

\caption{
	Comparison of different learning rates learning the \handmaxavg\ strategy
	when playing as the dealer
	over the course of one million games.
	}
\label{fig:expts-lr-comp}
\end{figure}


