
%%%
% Any results of the experimental runs
%%%

\subsection{Further Experiments}
\label{sec:findings-expts}

%%%
As the results of Round 2 emulated those of Round 1,
additional modifications were applied to the learning process.
%
These methods focused on directly affecting the weights applied to each
decision made,
both in learning and at choosing time.
%
The intended goal of these modifications were to find a policy
capable of learning a policy which plays the game consistently better than
random.
%
All modifications for these experiments started with the following common
parameters:
\begin{itemize}
	\item Scaling Factor: $s = 2.0$
	\item Decay: $d = 0.1$
	\item Random Exploration Chance: $e = 0.3$
\end{itemize}
%
Additionally,
the pegging records database was standardized across all experiments
as one selected from those resulting from the first round of training.
%
As a final note,
the mechanism for choosing a combination of cards using the \pvec\ vector 
during the tournament was altered slightly.
%
As previously mentioned,
when making an exploration step,
instead of using \pvec\ as a probability distribution,
the choice was made at uniform random,
in a manner true to $\epsilon$-greedy exploration~\cite{rl_book}.
%
This was done for the sake of simplicity.
%
Furthermore,
the difference in choosing mechanisms demonstrably
made no noticeable difference in the learning behavior,
perhaps as a result of \pvec\ being near-uniformly random for untrained states
and in essence a choice between three or four most desirable choices when
states had been visited.
%%%

%
%%%
% Explanation of round-robin structure and why it was done
%%%

\subsubsection*{Round-Robin Play}
\label{sec:findings-expts-rr}

%%%
Since the agents were deemed to be learning to outplay each other,
but not necessarily how to optimally play the game,
training against a single opponent for a million games was determined to be
an issue.
%
To determine how tailoring play to a single opponent affected the final outcome
weights,
a round-robin structure was implemented in which multiple agents would swap out
opponents occasionally.
%
Given a pool of potential agents,
every $N$ games,
a random pair of agents are selected from the pool and played against one
another.
%
By varying opponents every so often,
the intention of the round-robin structure should eliminate an agent's
dependence on another's strategy for performance.
%%%




%%%
% Discuss how neighboring weights were applied and what the results were
% TODO: results
%%%

\subsubsection*{Neighboring Weights}
\label{sec:findings-expts-neighbors}

%%%
% Process
%%%

%%%
In order to smooth out the strategy graphs
and prevent isolated states of separate weights,
a blending of neighboring weights was developed.
%
Rather than simply take the set of weights allocated to a single score location,
the agent instead takes a weighted average of all surrounding weight vectors
with its own location.
%
In other words,
\[
    \wvecm'_{p,o,d} = %\mathrm{l1norm}\left(
    \left|
    X\wvecm_{p,o,d} +
    Y \sum_{i\in\{-1,0,1\}} \sum_{j\in\{-1,0,1\}} w_{m+i,o+j,d}
    \right|_1
    %\right)
\]
where $X$, $Y$ are ratios of each vector's effect, $X+Y = 1$,
and $m+i$, $o+j \in [0,120]$.
%
The desired effect was to allow a score location to learn from its neighbors
so that a neighborhood effect was present in the decision.
%%%


\paragraph*{Results}

%%%
% Results:
%	Mostly same appearnace
%	fewer islands in hand_max_min territory for cribminavg/peggingminavggiven/
%		peggingmaxmedgained/peggingmaxavggained
%	Losing territory still in limbo, not much certainty there.
%%%

%%%
% Not really good at making a blending/gradient
% Definitely did eliminate islands though
%%%


\begin{figure}
\center

	\begin{subfigure}[t]{0.22\textwidth}
		\center
		\includegraphics[width=\stratgraphwidth]{images/findings/experiments/neighbors/strats/hand_max_min.png}
		\caption{\handmaxmin}
	\end{subfigure}
	~
	\begin{subfigure}[t]{0.22\textwidth}
		\center
		\includegraphics[width=\stratgraphwidth]{images/findings/experiments/neighbors/strats/hand_max_avg.png}
		\caption{\handmaxavg}
	\end{subfigure}
~
	\begin{subfigure}[t]{0.22\textwidth}
		\center
		\includegraphics[width=\stratgraphwidth]{images/findings/experiments/neighbors/strats/hand_max_med.png}
		\caption{\handmaxmed}
	\end{subfigure}
	~
	\begin{subfigure}[t]{0.22\textwidth}
		\center
		\includegraphics[width=\stratgraphwidth]{images/findings/experiments/neighbors/strats/hand_max_poss.png}
		\caption{\handmaxposs}
	\end{subfigure}

	\begin{subfigure}[t]{0.22\textwidth}
		\center
		\includegraphics[width=\stratgraphwidth]{images/findings/experiments/neighbors/strats/crib_min_avg.png}
		\caption{\cribminavg}
	\end{subfigure}
	~
	\begin{subfigure}[t]{0.22\textwidth}
		\center
		\includegraphics[width=\stratgraphwidth]{images/findings/experiments/neighbors/strats/pegging_max_avg_gained.png}
		\caption{\peggingmaxavggained}
	\end{subfigure}
~
	\begin{subfigure}[t]{0.22\textwidth}
		\center
		\includegraphics[width=\stratgraphwidth]{images/findings/experiments/neighbors/strats/pegging_max_med_gained.png}
		\caption{\peggingmaxmedgained}
	\end{subfigure}
	~
	\begin{subfigure}[t]{0.22\textwidth}
		\center
		\includegraphics[width=\stratgraphwidth]{images/findings/experiments/neighbors/strats/pegging_min_avg_given.png}
		\caption{\peggingminavggiven}
	\end{subfigure}

\caption{
	Final strategy graphs for an agent when playing as the dealer
	after being trained for 500,000 games
	by following a policy which uses a weighted sum of neighboring weights.
}
\label{fig:neighbor-strats}
\end{figure}


%%%
The agent was not capable of weighting the different strategies in a
gradient-like manner
by using the neighboring score locations' weights vectors.
%
In fact,
the \handmaxavg\ strategy,
which one would expect most to form a gradient with and around the
\handmaxmin\ strategy,
was perhaps the most negatively affected.
%
As seen in Figure~\ref{fig:neighbor-strats},
the \handmaxavg\ strategy graph has become more sparse in the winning
states (lower left)
with very little gray area surrounding the black.
%%%

%%%
Interestingly,
while the neighboring weights did not solidify any presence in the graphs,
the new training method did make progress in negative space.
%
This is to say that
the weighted neighbors learning method eliminated the usually present islands
of a single strategy within a swathe of another dominant strategy.
%
This white-space was present in both the winning and losing states.
%
This finding leads to the conclusion that weighted learning techniques
do indeed allow for a sharing of knowledge between like states,
but only insofar as to know what not to do.
%%%

%%%
Furthermore,
there is still a vast degree of uncertainty present in the losing states of
the strategy graphs.
%
The theory that the certainty of the winning states might potentially
bleed over into the losing side was not demonstrated over the course of
the first half million training games.
%
However,
since the elimination of islands shows an improvement in
preventing lucky happenstance to dictate a space's future,
the neighboring weights training method has shown usefulness in training a
cribbage agent in which states neighboring states are often similar in nature.
%%%




%%%
% Regularization to prevent too strong of weights
%%%

\subsubsection{Regularization}

%%%
% Process
%%%

%%%
As an attempt to prevent a single strategy's weight from being so strongly 
preferred that a second strategy could not hope to possibly gain ground,
a hard limit was placed on the pre-normalized update value.
%
Expressed mathematically,
\[
    w'_{m,o,d}[i] = \max\{K,cw_{m,o,d}[i]\}
\]
where $K$ is some constant value throughout the training.
%
While the value of $w_{m,o,d}[i]$ could exceed $K$ after re-normalization
for a particularly strongly weighted strategy,
the value could be seen converging to $K$ within a handful of iterations.
%
The desired intention of this regularization was to allow other strategies
the opportunity to overcome the bias of earlier strengthening of the strongest
strategy.
%%%


%%%
% Results:
%	Similar areas taken up for hand_max_{min,avg}
%		both share same space rather than Croatia/Boz&Herz-shape
%	Grayer, since values only so large
%	As max value is increased,
%		darker, more like un-regulated
%	Islands still present
%	losing, still uncertain
%		still chance of higher than regulation, because of aforementioned reason
%%%



%%%
% Discussion of how each strategy was used as a separate starting point
%%%

\subsubsection*{Policy Initialization}
\label{sec:findings-expts-starts}

%%%
Since a desired outcome of the learning process was to be able to use the
generated strategy graphs to tell how a hand should be played in a
certain score position,
a comparison was made between the produced agent and pure strategies
on a database of choices made by humans.
%
The website Daily Cribbage Hand
prompts its users with
a set of dealt cards,
their score,
their opponent's score,
and an indication of whether or not the player is the dealer for the round
\cite{dailycribbagehand}.
%
With this prompt,
the user then decides which cards they would
keep in the given scenario.
%%%

%%%
With access to recorded answers from this website,
the agent's choices could be compared
to how humans ranked the choice.
%to those made by humans in the same situation.
%
For each of the approximately 3600 usable records,
the choice the agent made was compared against those made by the users of the
website.
%%%

%%%
The results of this comparison,
seen in Table~\ref{tab:expts-starts-human}, % TODO: ref more tables if created
show that the agent trained in Round 2's losers' bracket chooses the same
set of cards as the human users only marginally more often than an agent with 
randomly allocated weights.
%
To approximately half of the prompts,
the trained agent chooses the same answer as most humans;
in almost 78 percent of the situations,
the answer given by the agent is within the top three most common human answers.
%
Additionally,
most pure strategies,
created by setting their weight to 1 while all other strategies are set to 0,
performed worse than the trained agent.
%
Notable exceptions to this trend are the \handmaxposs\ and \handmaxavg\ 
strategies,
suggesting that in more situations than the agent,
the typical human player will play according to what points can be expected
to be gained from the cut card.
%
Interestingly,
the \handmaxposs\ strategy's presence as the second most common pure strategy
used indicates a significant degree of risk-taking present in the users'
responses.
%
According to user comments,
this increased riskiness in play
may be a result of some users attempting to maximize points of the hand,
without the responsibility to actually play the resulting game,
and may not be entirely representative of actual in-game choices.
%%%

%%%
As a result of this finding,
some of these strategies were used as initial weights to the learning process
in order to determine if the agent could learn to fine-tune a policy starting
from a reasonable assertion of good game-play
as well as learn to discount demonstrably poor strategies.
%
Since the update mechanism for weights relies upon renormalization of a vector
which as been rewarded or punished,
no modifications would occur in the case of punishment of a completely pure strategy
since no other weights would have the chance to increase.
%
Therefore,
the pure strategies used before were slightly modified
such that each other
element of the \wvec\ vector would have a small initial value which could be
increased when this semi-pure strategy was punished.
%
Two agents,
each initialized with the same semi-pure strategy,
were played against each other for a million games,
with only a single agent updating its weights.
%%%


\begin{table}
	\centering

	\begin{tabular}{|r|c|c|c|p{4cm}|}
		\hline
		\textbf{Strategy} & \textbf{Top 1} & \textbf{Top 2} & \textbf{Top 3}
			& \textbf{Percentage in Top 3 Human Choices} \\
		\hline
		\peggingminavggiven & 160 & 303 & 458 & 12.64 \\\hline
		\peggingmaxmedgained & 268 & 519 & 796 & 21.97 \\\hline
		\peggingmaxavggained & 347 & 650 & 963 & 26.58 \\\hline
		\cribminavg & 380 & 177 & 1081 & 29.84 \\\hline
		\handmaxmin & 1576 & 2288 & 2666 & 73.59 \\\hline
		\textbf{Random} & \textbf{1581} & \textbf{2318} & \textbf{2759} &
			\textbf{76.15} \\\hline
		\handmaxmed & 1649 & 2353 & 2768 & 76.40 \\\hline
		\textbf{Trained} & \textbf{1706} & \textbf{2426} & \textbf{2821} &
			\textbf{77.86} \\\hline
		\handmaxposs & 1677 & 2433 & 2847 & 78.58 \\\hline
		\handmaxavg & 2066 & 2828 & 3168 & 87.44 \\\hline
	\end{tabular}
	% total = 3623
	\caption{
		Number of times the agent using a given strategy chose the same cards as
		the most common choice by human users
		according to 3623 total parsable records obtained from
		\cite{dailycribbagehand}.
		The columns labeled ``Top X'' display the number of times the given
		strategy's choice was within the top X choices of the user base.
		%if at least that many different choices were made
		In this table,
		\textbf{Random} is the best result from five agents which each used
		independently randomly allocated weights
		and
		\textbf{Trained} used an agent trained in Round 2's losers' bracket
		for one million games.
	}
	\label{tab:expts-starts-human}
\end{table}



\paragraph*{Results}

%%%%
%There are very two interesting trends that arise from starting from nearly pure
%strategies.
%%
%The first is that the starting strategy is,
%at least for those tested,
%the dominant strategy in winning positions.
%%
%The second is what learning occurs in losing positions.
%%%%

%%%
The comparison of different strategies' development after training
(Figure~\ref{fig:findings-expts-sanitycheck-matrix})
shows the agent learning applicable policies
for both winning and losing positions.
%
In the winning positions of the strategy graphs,
the starting strategy is further strengthened to dominance.
%
The purely mathematical operation of overcoming such an initially heavily
weighted strategy requires many games to explore enough
and perform better than the starting strategy
to be rewarded;
%
the explored combination of cards, itself,
may coincide with the starting strategy
already occupying these positions,
further increasing the difficulty.
%
Additionally,
since the opponent is a static agent always using the same starting
semi-pure strategy and never training these weights,
playing a similar strategy ensures a similar resulting position.
%
When already in the lead,
this is desirable as the agent will likely continue to be in a winning position
and closer to the goal score of 121.
%
An intriguing counter to this pattern of dominance in winning positions
is the \handmaxavg\ strategy
which yields some control of winning positions to \handmaxmed.
%
After normal training procedures,
this space is occupied by \handmaxmin,
which, characteristically, would not allow a larger gap to develop.
%not allowing a larger gap to develop.
%%%

%%%
In losing positions,
the agent develops a reasonable counter policy
to the given semi-pure opponent.
%
For instance,
when faced with \handmaxmin\ which will always play safe,
the agent learns to swing for the fences by playing according to \handmaxposs\ 
when it is losing
since the opponent will not be taking any risks itself
and risk is the only way to make up ground.
%
Similarly,
the agent learns that the best way to recover from a losing position
against \handmaxavg,
which plays to expectations and avoids unnecessary risk,
is to mostly play safely according to \handmaxmin.
%
As a counter to the dominance of a single strategy in the losing positions,
starting with \handmaxposs\ leads to a losing strategy
shared between \handmaxavg\ and \handmaxmed.
%
This is because either of these two strategies is likely to recover ground
against an agent which always tries to get the maximum amount of points
with no regard for the likelihood of this outcome.
%%%

% fig:findings-expts-sanitycheck-matrix
\begin{figure}
\centering

\begin{tabular}{ l l l l l l } %    c}
	%                    label
	%  l            | l1   l2   l3   l4
	%  a   ---------+-------------------
	%  b   starting | a1 | a2 | a3 | a4
	%  e   strat    | b1 | b2 | b3 | b4
	%  l   ...
	& & \multicolumn{4}{c}{\textit{Strategy}} \\
	& & \handmaxmin & \handmaxavg & \handmaxposs & \handmaxmed \\
	%\cline{3-6}
	\multirow{7}{*}{
	\rotatebox{90}{
	\parbox[c]{6.5cm}{
		\textit{Starting Strategy}
	}
	}
	}
	%
	& \rotatebox[origin=c]{90}{\handmaxmin}
		&\parbox[c]{1em}{\includegraphics[width=2.5cm]{images/findings/experiments/starting_points/matrix_handmaxmin_handmaxmin.png}}
		&\parbox[c]{1em}{\includegraphics[width=2.5cm]{images/findings/experiments/starting_points/matrix_handmaxmin_handmaxavg.png}}
		&\parbox[c]{1em}{\includegraphics[width=2.5cm]{images/findings/experiments/starting_points/matrix_handmaxmin_handmaxposs.png}}
		&\parbox[c]{1em}{\includegraphics[width=2.5cm]{images/findings/experiments/starting_points/matrix_handmaxmin_handmaxmed.png}}
		%&
	\\ & & & & & \\
	%
	& \rotatebox[origin=c]{90}{\handmaxavg}
		&\parbox[c]{1em}{\includegraphics[width=2.5cm]{images/findings/experiments/starting_points/matrix_handmaxavg_handmaxmin.png}}
		&\parbox[c]{1em}{\includegraphics[width=2.5cm]{images/findings/experiments/starting_points/matrix_handmaxavg_handmaxavg.png}}
		&\parbox[c]{1em}{\includegraphics[width=2.5cm]{images/findings/experiments/starting_points/matrix_handmaxavg_handmaxposs.png}}
		&\parbox[c]{1em}{\includegraphics[width=2.5cm]{images/findings/experiments/starting_points/matrix_handmaxavg_handmaxmed.png}}
		%&\parbox[c]{1em}
	\\& & & & & \\
	%
	& \rotatebox[origin=c]{90}{\handmaxposs}
		&\parbox[c]{1em}{\includegraphics[width=2.5cm]{images/findings/experiments/starting_points/matrix_handmaxposs_handmaxmin.png}}
		&\parbox[c]{1em}{\includegraphics[width=2.5cm]{images/findings/experiments/starting_points/matrix_handmaxposs_handmaxavg.png}}
		&\parbox[c]{1em}{\includegraphics[width=2.5cm]{images/findings/experiments/starting_points/matrix_handmaxposs_handmaxposs.png}}
		&\parbox[c]{1em}{\includegraphics[width=2.5cm]{images/findings/experiments/starting_points/matrix_handmaxposs_handmaxmed.png}}
		%&\parbox[c]{1em}
	\\& & & & & \\
	%
	& \rotatebox[origin=c]{90}{\handmaxmed}
		&\parbox[c]{1em}{\includegraphics[width=2.5cm]{images/findings/experiments/starting_points/matrix_handmaxmed_handmaxmin.png}}
		&\parbox[c]{1em}{\includegraphics[width=2.5cm]{images/findings/experiments/starting_points/matrix_handmaxmed_handmaxavg.png}}
		&\parbox[c]{1em}{\includegraphics[width=2.5cm]{images/findings/experiments/starting_points/matrix_handmaxmed_handmaxposs.png}}
		&\parbox[c]{1em}{\includegraphics[width=2.5cm]{images/findings/experiments/starting_points/matrix_handmaxmed_handmaxmed.png}}
		%&\parbox[c]{1em}
	\\
\end{tabular}

\caption{}
\label{fig:findings-expts-sanitycheck-matrix}
\end{figure}


%%%
As this experiment showed promise in an agent potentially learning
to out-play its opponent,
an agent trained with a beginning pure strategy of \handmaxavg\ was played
against its previous iterations in several 10,000-game tournaments.
%
The results,
depicted in Figure~\ref{fig:expts-sanitycheck-spreads}
show a steady decrease in performance as training proceeds.
%
Earlier iterations,
closer to the semi-pure \handmaxavg\ strategy,
perform better than the trained agents.
%
Without further context,
it would appear that \handmaxavg\ is simply a good strategy which is difficult
enough to overcome,
so the training framework is forcing an adaptation
unnecessarily and undesirably.
%
However,
as the strategy graphs developed against a static semi-pure \handmaxavg\ 
opponent match those developed against a random agent
(Figure~\ref{fig:expts-sanitycheck-strats}),
this conclusion is unfounded.
%
If the \handmaxavg\ strategy were clearly superior to others,
then very little modification would be made by the agent.
%
Instead,
the losing states still drifted away from the starting strategy.
%
The only conclusions that can be made from this data are that
losing positions are fundamentally difficult to recover from
and that
cribbage involves much more than is encapsulated by the small set of
strategies with which the agent has been programmed.
%%%

% fig:expts-sanitycheck-spreads[-{a,b}]
% Point spreads for two runs of blah

\begin{figure}
\center

% TODO: actual images

\begin{subfigure}[b]{0.45\textwidth}
	%\includegraphics[width=\linewidth]{images/findings/experiments/learning_rate/tourny_a.png}
	\caption{A trained agent plays against previous iterations of itself.}
	\label{fig:expts-sanitycheck-spreads-a}
\end{subfigure}
~
\begin{subfigure}[b]{0.45\textwidth}
	%\includegraphics[width=\linewidth]{images/findings/experiments/learning_rate/tourny_b.png}
	\caption{
		An agent using a semi-pure \handmaxavg\ strategy
		plays the epoch checkpoints of an agent which has been trained
		against this strategy.
	}
	\label{fig:expts-sanitycheck-spreads-b}
\end{subfigure}

\caption{
	Point spreads of several 10,000-game tournaments between agents of varying
	training levels when started with set of semi-pure \handmaxavg\ strategy
	weights.
}
\label{fig:expts-sanitycheck-spreads}
\end{figure}


%%%%%%
%%%Furthermore,
%%%as the agents were previously each trained against different pure strategies,
%%%the ability to improve a basic policy to overcome a randomly-weighted agent,
%%%rather than to learn a policy from scratch,
%%%was tested by training an agent starting with a semi-pure
%%%\handmaxavg\ strategy against an unlearning agent with random weights.
%%%%
%%%The results of this further training,
%%%seen in Figure~\ref{fig:expts-sanitycheck-strats},
%%%show that the same behavioral trends are learned by an
%%%agent starting with a semi-pure \handmaxavg\ strategy 
%%%when the opponent is random
%%%as well as when the opponent follows the same semi-pure strategy.
%%%%%%

% fig:expts-sanitycheck-strats

\begin{figure}
\center

% TODO: actual images

	\begin{subfigure}[t]{0.22\textwidth}
		%\includegraphics[width=\textwidth]{images/findings/experiments/starting_points/hand_max_min.png}
		\caption{\handmaxmin}
	\end{subfigure}
	~
	\begin{subfigure}[t]{0.22\textwidth}
		%\includegraphics[width=\textwidth]{images/findings/experiments/starting_points/hand_max_avg.png}
		\caption{\handmaxavg}
	\end{subfigure}
	~
	\begin{subfigure}[t]{0.22\textwidth}
		%\includegraphics[width=\textwidth]{images/findings/experiments/starting_points/hand_max_med.png}
		\caption{\handmaxmed}
	\end{subfigure}
	~
	\begin{subfigure}[t]{0.22\textwidth}
		%\includegraphics[width=\textwidth]{images/findings/experiments/starting_points/hand_max_poss.png}
		\caption{\handmaxposs}
	\end{subfigure}

	\begin{subfigure}[t]{0.22\textwidth}
		%\includegraphics[width=\textwidth]{images/findings/experiments/starting_points/crib_min_avg.png}
		\caption{\cribminavg}
	\end{subfigure}
	~
	\begin{subfigure}[t]{0.22\textwidth}
		%\includegraphics[width=\textwidth]{images/findings/experiments/starting_points/pegging_max_avg_gained.png}
		\caption{\peggingmaxavggained}
	\end{subfigure}
	~
	\begin{subfigure}[t]{0.22\textwidth}
		%\includegraphics[width=\textwidth]{images/findings/experiments/starting_points/pegging_max_med_gained.png}
		\caption{\peggingmaxmedgained}
	\end{subfigure}
	~
	\begin{subfigure}[t]{0.22\textwidth}
		%\includegraphics[width=\textwidth]{images/findings/experiments/starting_points/pegging_min_avg_given.png}
		\caption{\peggingminavggiven}
	\end{subfigure}

\caption{
	All final strategy strengths for an agent
	which started with a 70\% semi-pure \handmaxavg\ strategy
	and was trained against an agent with unchanging random weights
	for one million games
	when playing as the dealer.
}
\label{fig:expts-sanitycheck-strats}
\end{figure}





\subsubsection*{Punishment Severity}
\label{sec:findings-expts-punishments}

%%%
Under the assumption that a player in a losing position
early in the game tended to never recover and thus lost as a result of
unfortunate positioning,
it followed that punishing losing states for something beyond the agent's
control was potentially unfair.
%
Furthermore,
it was postulated that since the punishment mechanism, in effect, cycled
strategies
and that there was a possibility that an occasionally winning strategy was often
outweighed by the tendency to lose,
a less strict method of punishment could be used to ensure that the occasional
win from a losing position remains visible.
%
As such,
the update step for modifying the weights was adjusted slightly
so that the constant adjustment factor was significantly smaller for losing
games than it was for winning games.
%
Instead of using the adjustment constant of
$C = s \cdot (\textit{PlayerScore} - \textit{OpponentScore})$
for both winning and losing agents,
the losing agent's adjustment factor was defined as
$C = \frac{1}{4} s \cdot (\textit{PlayerScore} - \textit{OpponentScore})$.
%
The reduction to $\frac{1}{4}$ was made arbitrarily for illustrative purposes.
%%%

\paragraph*{Results}

%%%
The introduction of an amount of forgiveness did not lead to any worthwhile
difference in learned policy.
%
In contrast to the goal of allowing an occasionally good strategy to form
in losing positions,
those positions are even less sure as to which strategy to take,
as seen in Figure~\ref{fig:findings-expts-punish-strats}.
%
Therefore,
it may be concluded that the increased likelihood of losing is 
likely not unfairly punishing potentially good recovery policies.
%
It can be argued that $\frac{1}{4}$ was not a proper ratio
to compensate for the likelihood of loss,
but the similarity of patterns learned and decreased certainty
indicate that the punishment mechanism is functioning adequately.
%
Similarly,
while reducing the amplitude of changes
does lead to a more evenly mixed losing policy,
this is effectively the result of a learning rate adjustment.
%
As will be shown in the learning rate experiments,
these forms of adjustments do not lead to differences in learned behaviors.
%%%

% fig:findings-expts-punish-strats

\begin{figure}
\center

	\begin{subfigure}[t]{0.22\textwidth}
		\includegraphics[width=\stratgraphwidth]{images/findings/experiments/punishment/strategies_handmaxmin.png}
		\caption{\handmaxmin}
	\end{subfigure}
	~
	\begin{subfigure}[t]{0.22\textwidth}
		\includegraphics[width=\stratgraphwidth]{images/findings/experiments/punishment/strategies_handmaxavg.png}
		\caption{\handmaxavg}
	\end{subfigure}
	~
	\begin{subfigure}[t]{0.22\textwidth}
		\includegraphics[width=\stratgraphwidth]{images/findings/experiments/punishment/strategies_handmaxmed.png}
		\caption{\handmaxmed}
	\end{subfigure}
	~
	\begin{subfigure}[t]{0.22\textwidth}
		\includegraphics[width=\stratgraphwidth]{images/findings/experiments/punishment/strategies_handmaxposs.png}
		\caption{\handmaxposs}
	\end{subfigure}

	\begin{subfigure}[t]{0.22\textwidth}
		\includegraphics[width=\stratgraphwidth]{images/findings/experiments/punishment/strategies_cribminavg.png}
		\caption{\cribminavg}
	\end{subfigure}
	~
	\begin{subfigure}[t]{0.22\textwidth}
		\includegraphics[width=\stratgraphwidth]{images/findings/experiments/punishment/strategies_peggingmaxavggained.png}
		\caption{\peggingmaxavggained}
	\end{subfigure}
	~
	\begin{subfigure}[t]{0.22\textwidth}
		\includegraphics[width=\stratgraphwidth]{images/findings/experiments/punishment/strategies_peggingmaxmedgained.png}
		\caption{\peggingmaxmedgained}
	\end{subfigure}
	~
	\begin{subfigure}[t]{0.22\textwidth}
		\includegraphics[width=\stratgraphwidth]{images/findings/experiments/punishment/strategies_peggingminavggiven.png}
		\caption{\peggingminavggiven}
	\end{subfigure}

\caption{
	Final strategy graphs for an agent which has less severe punishment
	after training for one million games.
}
\label{fig:findings-expts-punish-strats}
\end{figure}





\subsubsection*{Learning Rate Adjustment}
\label{sec:findings-expts-learnrate}

%%%
In order to determine if the learning rate was too high in Round 2,
even though it had been significantly reduced from Round 1,
a varying amount of learning rates were tried.
%
These runs were intended to see if an optimal policy was being overstepped by
making too large of an adjustment.
%%%


\paragraph*{Results}

%%%
% 2 possibilities foreseeable for results:
%	1. patterns are the same
%	2. something new is learned
%%%

%%%
Even with very reduced scaling factors,
no difference in behavioral trends learned
was observed.
%
However,
it has been demonstrated in Figure~\ref{expts-lr-comp} that
a decrease in scaling factor leads to slower adjustments,
showing that the scaling factor does indeed function as a learning rate.
%
In fact,
the images along each counterdiagonal
are nearly identical
because they roughly align in
cumulative adjustment magnitude made to the weights.
%
Therefore,
it is safe to conclude that some optimum is not being stepped over,
allowing a worse set of weights to be pursued instead.
%%%

% fig:expts-lr-comp
\begin{figure}[h]
	\centering

	\begin{tabular}{c | c c c c}
		% Outline:
		%   s\g |  250k | 500k | 750k | 1mm
		%	0.25
		%   0.50
		%   1.00
		%   1.50
		$s$\textbackslash game & 250,000 & 500,000 & 750,000 & 1,000,000 \\
		\hline
		0.25 & % a & b & c & d
			\includegraphics[width=2cm]{images/findings/experiments/learning_rate/lr_025_250.png} & % 250
			\includegraphics[width=2cm]{images/findings/experiments/learning_rate/lr_025_500.png} & % 500
			\includegraphics[width=2cm]{images/findings/experiments/learning_rate/lr_025_750.png} & % 750
			\includegraphics[width=2cm]{images/findings/experiments/learning_rate/lr_025_1mm.png} \\ % 1mm
		0.50 & 
			\includegraphics[width=2cm]{images/findings/experiments/learning_rate/lr_050_250.png} & % 250
			\includegraphics[width=2cm]{images/findings/experiments/learning_rate/lr_050_500.png} & % 500
			\includegraphics[width=2cm]{images/findings/experiments/learning_rate/lr_050_750.png} & % 750
			\includegraphics[width=2cm]{images/findings/experiments/learning_rate/lr_050_1mm.png} \\ % 1mm
		1.00 & 
			\includegraphics[width=2cm]{images/findings/experiments/learning_rate/lr_100_250.png} & % 250
			\includegraphics[width=2cm]{images/findings/experiments/learning_rate/lr_100_500.png} & % 500
			\includegraphics[width=2cm]{images/findings/experiments/learning_rate/lr_100_750.png} & % 750
			\includegraphics[width=2cm]{images/findings/experiments/learning_rate/lr_100_1mm.png} \\ % 1mm
		1.50 & 
			\includegraphics[width=2cm]{images/findings/experiments/learning_rate/lr_150_250.png} & % 250
			\includegraphics[width=2cm]{images/findings/experiments/learning_rate/lr_150_500.png} & % 500
			\includegraphics[width=2cm]{images/findings/experiments/learning_rate/lr_150_750.png} & % 750
			\includegraphics[width=2cm]{images/findings/experiments/learning_rate/lr_150_1mm.png} \\ % 1mm
	\end{tabular}

\caption{
	Comparison of different learning rates learning the \handmaxavg\ strategy
	when playing as the dealer
	over the course of one million games.
	}
\label{fig:expts-lr-comp}
\end{figure}





\subsubsection*{Decay Rate}
\label{sec:findings-expts-decay}

%%%
In addition to the learning rate,
the decay parameter $d$ was also adjusted to see what sort of effect it
would have on learning.
%
Instead of the default decay rate of 10\%,
rates ranging from 0\% to 50\%
were tested to demonstrate the effect of temporal difference learning.
%%%

\paragraph*{Results}

%%%
Adjustments in decay rate made no change in the behavioral trends learned.
%
However,
as can be seen in Figure~\ref{fig:expts-decay-comp},
the rate of learning of these behaviors in earlier states is highly affected
by this parameter.
%
With lower decay rates,
more of the responsibility for the final result of the game is transferred
to earlier game states,
meaning trends develop quickly.
%
Contrarily,
with higher decay rates,
it is nearly impossible to learn earlier game states,
so behavior remains random,
as clearly seen when $d = 0.50$.
%
For explanation,
a typical cribbage game will take approximately ten hands
to complete.
%
At this rate,
the first states will be adjusted by a factor of
$(1-d)^{10} = (0.5)^{10} \approx 0.00098$.
%
Even over the course of one million games,
this small magnitude of modification,
combined with the rate of visitation of a single state,
is not enough for any meaningful learning to occur
in these earlier states.
%%%

% fig:expts-decay-comp
\begin{figure}[h]
	\centering

	\begin{tabular}{c | l l l l}
		% Outline:
		%   s\g |  250k | 500k | 750k | 1mm
		%   0.00
		%   0.10
		%   0.25
		%   0.50
		$d$ \textbf{\textbackslash} game & 250,000 & 500,000 & 750,000 & 1,000,000 \\
		\hline
		\\
		0.00 &
			\parbox[c]{5em}{\includegraphics[width=\stratgraphwidthsmall]{images/findings/experiments/decay/decay_000_250.png}} & % 250
			\parbox[c]{5em}{\includegraphics[width=\stratgraphwidthsmall]{images/findings/experiments/decay/decay_000_500.png}} & % 500
			\parbox[c]{5em}{\includegraphics[width=\stratgraphwidthsmall]{images/findings/experiments/decay/decay_000_750.png}} & % 750
			\parbox[c]{5em}{\includegraphics[width=\stratgraphwidthsmall]{images/findings/experiments/decay/decay_000_1mm.png}} \\ % 1mm
		\\
		0.10 & 
			\parbox[c]{5em}{\includegraphics[width=\stratgraphwidthsmall]{images/findings/experiments/decay/decay_010_250.png}} & % 250
			\parbox[c]{5em}{\includegraphics[width=\stratgraphwidthsmall]{images/findings/experiments/decay/decay_010_500.png}} & % 500
			\parbox[c]{5em}{\includegraphics[width=\stratgraphwidthsmall]{images/findings/experiments/decay/decay_010_750.png}} & % 750
			\parbox[c]{5em}{\includegraphics[width=\stratgraphwidthsmall]{images/findings/experiments/decay/decay_010_1mm.png}} \\ % 1mm
		\\
		0.25 & 
			\parbox[c]{5em}{\includegraphics[width=\stratgraphwidthsmall]{images/findings/experiments/decay/decay_025_250.png}} & % 250
			\parbox[c]{5em}{\includegraphics[width=\stratgraphwidthsmall]{images/findings/experiments/decay/decay_025_500.png}} & % 500
			\parbox[c]{5em}{\includegraphics[width=\stratgraphwidthsmall]{images/findings/experiments/decay/decay_025_750.png}} & % 750
			\parbox[c]{5em}{\includegraphics[width=\stratgraphwidthsmall]{images/findings/experiments/decay/decay_025_1mm.png}} \\ % 1mm
		\\
		0.50 & 
			\parbox[c]{5em}{\includegraphics[width=\stratgraphwidthsmall]{images/findings/experiments/decay/decay_050_250.png}} & % 250
			\parbox[c]{5em}{\includegraphics[width=\stratgraphwidthsmall]{images/findings/experiments/decay/decay_050_500.png}} & % 500
			\parbox[c]{5em}{\includegraphics[width=\stratgraphwidthsmall]{images/findings/experiments/decay/decay_050_750.png}} & % 750
			\parbox[c]{5em}{\includegraphics[width=\stratgraphwidthsmall]{images/findings/experiments/decay/decay_050_1mm.png}} \\ % 1mm
	\end{tabular}

\caption{
	Comparison of different decay rates learning the \handmaxavg\ strategy
	when playing as the dealer
	over the course of one million games.
	}
\label{fig:expts-decay-comp}
\end{figure}



