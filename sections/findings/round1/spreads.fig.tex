% Point spreads for two runs of blah

\begin{figure}
\center

\begin{subfigure}[b]{0.45\textwidth}
	\includegraphics[width=\linewidth]{images/findings/round1/spreads.png}
	\caption{An agent plays against previous iterations of itself.}
	\label{fig_r1-spreads_a}
\end{subfigure}
~
\begin{subfigure}[b]{0.45\textwidth}
	\includegraphics[width=\linewidth]{images/findings/round1/random-spreads.png}
	\caption{A random agent plays against later, more learned agents.}
	\label{fig_r1-spreads_b}
\end{subfigure}

%\includegraphics[width=0.9\textwidth]{images/findings/round1/spreads.png}

\caption{
	Point spreads across twenty 100-game tournaments pitting a winning
	agent against its checkpoints.
	Here, a positive point spread indicates that the fully-trained agent has
	accumulated more points than its opponent,
	an agent created from a checkpoint generated after the number of training
	game epochs indicated on the x-axis.
}
\label{fig_r1-spreads}
\end{figure}
