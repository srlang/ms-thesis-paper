% Table to represent the tournament information

\begin{table}
\center
\begin{tabular}{|r|c|c|}
	\hline
	\textbf{Game} & \textbf{Random} & \textbf{Trained}\\\hline

	1 & 104 & 121 \\\hline
	2 & 121 & 114 \\\hline
	3 &  75 & 121 \\\hline
	4 & 121 & 118 \\\hline
	5 & 111 & 121 \\\hline
	6 & 121 &  99 \\\hline
	7 & 121 &  87 \\\hline
	8 & 110 & 121 \\\hline
	9 & 121 &  64 \\\hline

\end{tabular}
~
\begin{tabular}{|r|c|c|c|}
	\hline
	\textbf{Agent} & \textbf{Score} & \textbf{Point Spread} & \textbf{Wins}
	\\\hline
	Random & 12 & +39 & 5
	\\\hline
	Trained  & 9 & \textemdash & 4
	\\\hline
\end{tabular}

\caption{
	Results of a nine-game tournament played between
	a randomly-weighted agent
	and
	a trained agent after learning for one million games.
	%
	The table on the left shows the final point amounts for each of the nine
	games.
	%
	The table on the right shows the score of each player,
	as determined by the ACC scoring rules for match play;
	the total point spread for the top player,
	omitted for the second player for readability
	as it would just be the negative of the first player's;
	and the number of times each player won a game.
}

\label{tab_r1-randtourny}

\end{table}
