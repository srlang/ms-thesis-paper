% Figure for the flipbook of strategies over time

\begin{figure}
\center

	\begin{subfigure}[b]{0.4\textwidth}
	\includegraphics[width=\linewidth]{images/findings/round2/flipbook/random/checkpoint_000000.png}
	\caption{Starting Weights}
	\end{subfigure}
	~
	\begin{subfigure}[b]{0.4\textwidth}
	\includegraphics[width=\linewidth]{images/findings/round2/flipbook/random/checkpoint_200000.png}
	\caption{After 200,000 games played}
	\end{subfigure}

	\begin{subfigure}[b]{0.4\textwidth}
	\includegraphics[width=\linewidth]{images/findings/round2/flipbook/random/checkpoint_400000.png}
	\caption{After 400,000 games played}
	\end{subfigure}
	~
	\begin{subfigure}[b]{0.4\textwidth}
	\includegraphics[width=\linewidth]{images/findings/round2/flipbook/random/checkpoint_600000.png}
	\caption{After 600,000 games played}
	\end{subfigure}

	\begin{subfigure}[b]{0.4\textwidth}
	\includegraphics[width=\linewidth]{images/findings/round2/flipbook/random/checkpoint_800000.png}
	\caption{After 800,000 games played}
	\end{subfigure}
	~
	\begin{subfigure}[b]{0.4\textwidth}
	\includegraphics[width=\linewidth]{images/findings/round2/flipbook/random/checkpoint_999999.png}
	\caption{Final Weights}
	\end{subfigure}

\caption{
	Training weights representation for a winner bracket agent's \handmaxavg\
	strategy when the agent is the dealer
	over the course of the one million games of Round 2.
	Training weights representation for an agents \handmaxavg\ strategy
	over the course of one million games
	in which the opponent was always played using randomly allocatd weights.
}
\label{fig:r2-flip-random}
\end{figure}
