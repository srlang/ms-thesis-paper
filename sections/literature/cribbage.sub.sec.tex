
\subsection{Prior Cribbage Research}

%%%
% Prior research performed in cribbage
%	paper on dealer's advantage
%	Evolutionary/genetic card picker
%	O'Connor's MLP
%	whatever else found during proposal time
%%%

%%%
On the topic of cribbage very little research has been done,
perhaps as a result of its relative lack of popularity to such games as poker
or its stochastic environment being too large to tackle.
%
There are three main papers in which cribbage receives the main focus:
a mathematical analysis of dealer advantage~\cite{cribbage_optimal_ev},
a genetic learner applied to the discard phase~\cite{adaptive_cribbage},
and a multilayer perceptron attempt to apply a simpler version of TD($\lambda$)
to cribbage~\cite{roconnor_cs486}.
%%%

%%%
In his thesis,
Philip Martin set out to find if the player which begins as the dealer
had any statistical advantage in winning
\cite{cribbage_optimal_ev}.
%
The method used to accomplish this task was to enumerate and evaluate all
possible combinations of cards which can be dealt to a player.
%
From there,
a matrix of potential average crib scores was created as a table indexed on each
axis by one of ${52 \choose 2}$ combinations,
introducing the minor accuracy of forgetting which cards have been seen by each
player from their respective hands.
%
Using the bounds found,
certain basic strategies were proven to be suboptimal,
e.g. avoiding throwing a pair or fifteen to the crib as the pone,
since their at least one of their expected outcomes
was found to be below the lower bound for optimal strategies.
%
Despite the slight inaccuracy and assumption of a uniform distribution,
the paper did conclude that being the first player to act as dealer gave the
player an expected advantage of approximately 5 points on average.
%
However,
since different strategies needed for other portions of the game were not
mentioned,
by the author's own admission this study barely scratched the surface of
research into the game.
%%%

%%%
The next,
and perhaps most useful,
study performed on the game of cribbage is Robert O'Connor's undergraduate
projecti~\cite{roconnor_cs486}
on adapting TD($\lambda$)~\cite{tdgammon} to the domain of cribbage.
%
O'Connor trained a feedforward multilayer perceptron with only a single hidden
layer to play a single hand of cribbage.
%
He used TD($\lambda$) to adjust the weights as training proceeded with the
network playing full games against itself:
no score context data was included in the input vector for training.
%
After training for millions of games,
the agent was able to choose hands reliably better than random,
but just shy of an algorithm which would choose by maximum expected outcome.
%%%

