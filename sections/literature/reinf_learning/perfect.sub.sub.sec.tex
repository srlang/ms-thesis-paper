
%%%
% Discussion of DeepMind's AlphaGo strategy and DeepBlue's Chess
%%%

\subsubsection*{Perfect Information Games}

%%%
Perfect information games are those in which the entire game state is observable
at a given time~\cite{perfinfo}.
%
Furthermore,
an action has a definite and entirely predictable outcome:
e.g. moving a chess piece on the board in a specific arrangement will always
lead to the same resulting arrangement.
%
Checkers, chess, and Go are all examples of perfect information games.
%%%

%%%
Jumping ahead by nearly 40 years,
the basic ideas of Samuel's paper can be seen applied to the slightly different
domain of chess with IBM's DeepBlue
and its victory against Garry Kasparov.
%
The DeepBlue team greatly expanded upon the idea of a minimax search to create a
massively parallel search system for chess evaluation
\cite{ibm_deepblue}.
%
The final system was able to evaluate hundreds of millions of chess positions
per second,
millions of times faster than Samuel's capabilities at the time.
%
However,
the speed in evaluation was a result of hard-wiring an evaluation function at
the chip level using chips custom-built for the purpose.
%
As with Rote learning,
very little in terms of actual learning how to play or adjustments to internal
mechanisms occurred
beyond an expanded search space made possible by more capable hardware.
%
Despite this philosophical gripe,
the IBM team was able to outplay a Grandmaster of chess,
winning the match four games to two.
%%%

%%%
% deepmind's alphago
%%%

%%%
Of notable attention recently is Google DeepMind's
success in the game of Go with AlphaGo
\cite{deepmind_alphago}.
%
More satisfying than DeepBlue's search approach,
AlphaGo used multiple neural networks to evaluate value and policy functions.
%
A first neural network was trained using a database of human-played games
to predict human play for a given board position.
%
Using this network as a starting point,
a policy network was trained
along with a value network which worked together to determine the optimal next
move for the agent.
%
Both of these previous networks were trained through self-play.
%
The network resulting from this self-play was able to beat 
European Go Champion Fan Hui in match play
five games to one,
eventually going as far as winning against
Grandmaster Lee Sedol four games to one~\cite{deepmind_overview}.
%%%

%%%
% TODO: more detail on alphago structure
%%%

%%%
DeepMind then took a step further,
%following in Tesauro's footsteps~\cite{tdgammon},
removing all human-imposed knowledge and
training a single neural network to play \textit{tabula rasa}, i.e. from scratch.
%
Using only this neural network and a tree search algorithm,
AlphaGo Zero, the new program,
was able to defeat the previous AlphaGo 89 games to 11
\cite{deepmind_alphago_zero}.
%%%

%%%
% TODO: more about alphago zero, surely
%%%

