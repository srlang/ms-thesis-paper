\subsection{Reinforcement Learning}

%%%
% Background of topics in reinforcement learning
%%%

%%%
% What is Reinforcement Learning:
% When is it used:
%%%

%%%
Reinforcement learning is the machine learning equivalent of learning from one's
failures,
rather than being coached to the correct answer.
%
In classical machine learning methodologies,
the agent discovers an optimum model for a problem by approximation methods
centered around minimizing a set loss function over a given set of data
while assuming a known model for the solution.
% ^^^ TODO: check if accurate enough to say
%
In reinforcement learning,
however,
the agent finds the optimal solution to the problem by repeatedly taking an
action in an environment and gaining a reward or punishment for each
action taken.
%
It is the same principle used in teaching a pet or animal to do a trick:
offer a full or partial reward for successful completion of the trick or
for progress in the correct direction.
%
As a comparison to teaching a human how to add,
the strategy used in classical machine learning would be what is used in
classrooms today:
teach the method of adding digits and handling carry-over,
giving some guidance and sample problems to ensure the technique is solidly
replicable.
%
Meanwhile, teaching a human how to add by reinforcement learning would
mean merely quizzing the subject by asking him to answer an addition problem
while giving a vague hint as to how right or wrong they were.
%
After enough rounds of this,
the student will eventually figure out his own method for adding two numbers
with the same accuracy as an established method.
%%%

%%%
While this may seem like a silly example where classical methods would clearly
be the superior method,
where reinforcement learning comes into its own is in situations in which no
known answer exists for a problem.
%
Take for instance the problem of learning chess.
%
In humans
basic strategies for how to handle certain situations can be taught,
but these are all from one's own or others' prior experience,
and while they may bolster knowledge on heretofore unknown situations,
they cannot possibly cover all possible chessboard layouts.
%
The best way to learn, in humans and computers, is by doing.
%
Barring savants such as Bobby Fischer,
humans will learn how to play chess better by playing games against varying
opponents.
%
After a game has been played,
the player can see what worked and failed during the game to cause the win or
loss and extrapolate what to do if a similar situation occurs.
%
Classical teaching methodologies would not be highly applicable in this
situation because there is no single correct strategy in chess that can
be taught.
%%%

%%%
% Loss functions in 'classical' machine learning
%	and how they are unusable in reinforcement learning scenarios
%	What is a reinf learning scenario?
%		when is it used/favored
%%%

\subsubsection{Rewards and the Environment}

%%%
% Explanation of basic background topics in reinforcement learning
%	what is the environment
%	what is reward defined as
%	what is a policy,goal,etc.
%%%

%%%
The three most important, constantly interacting components in a reinforcement
learning scenario are
the environment, the agent, and rewards.
%
The agent must learn to navigate the environment in order to maximize its
rewards,
much like a mouse navigating a maze to retrieve the cheese at the end.
%%%

\paragraph{The Environment}

%%%
In reinforcement learning scenarios,
the agent interacts with and navigates what is known as the environment.
%
What exactly constitutes the environment is problem-specific and not often
clearly defined.
%
% TODO: vvv check if "However, ..." is allowed in English grammar
However, it can be expressed as a set of states in which one or more actions
can be taken.
%
A state can be any situation in which the agent finds itself.
%
In chess this can be a specific board arrangement.
%
In golf, this would be a location somewhere along the course of play and how
much wind is blowing in which direction.
%
An action is an interaction the agent can make with the environment to alter
its current position.
%
In the example of chess,
an action is discrete and would be to move a piece X to position Y,
e.g. moving the bishop to G4.
%
In golf, the action is much more continuous and may be
which club to use in which direction and with how much power.
%%%

\paragraph{Rewards and Goals}

\paragraph{Policies}


\subsubsection{Learning an Optimal Policy}

%%%
% A walk-through of policy evaluation, policy improvement, and value iteration
%%%

%%%
%For the sake of discussing how an optimal policy can be learned,
%it is useful from hereon to limit the scope of discussion to the case
%of a discrete set of states and actions.
%
Although applicable to both discrete and continuous state representations,
it is useful for the sake of illustration to limit the scope of discussion to
discrete representations.
%
Heretofore, unless otherwise stated, all discussion will assume a discrete
representation.
%%%

% TODO: need a value function explanation (background knowledge, maybe?)

\paragraph{Metrics}

%%%
% Discuss what it means to be optimal/ pi_i > pi_j
% Explanation of value of a state: v_{pi}(s)
% Quality of an action q_{pi}(s,a)
%%%

%%%
A state can have a \textit{value} associated with it given a policy $\policy$.
%
The value of a state $s \in \States$ under policy $\policy$ is denoted
$v_\policy(s)$
which can signal the ``worth'' of the given state
and is defined as the expected total reward by following policy $\policy$ from
state $s$:
\begin{align*}
	v_\policy(s) &= \Expectation_\policy\left[G_t | S_t = s\right] \\
		&= \Expectation_\policy \left[
				\sum_{k=0}^{\infty} \gamma^k R_{t+k+1} | S_t = s
			\right]
\end{align*}
%
The optimum value of a state $v_*(s)$ is defined as:
\[ v_*(s) = \max_\policy v_\policy(s)\ \forall s \in \States \]
%%%

%%%
Similarly,
an action can have its own ``worth'' or \textit{quality} assigned to it under a
specific policy.
%
The quality of an action $a\in\Actions$ at state $s\in\States$ under policy
$\policy$ is defined as:
\begin{align*}
	q_\policy(s,a)
		&= \Expectation_\policy\left[ G_t | S_t = s, A_t = a \right]
		\\
		&= \Expectation_\policy\left[
				\sum_{k=0}^{\infty}\gamma^k R_{t+k+1} | S_t = s, A_t = a
			\right]
\end{align*}
and its optimum $q_*(s,a)$ is accordingly defined as:
\[
	q_*(s,a) = \max_{\policy}q_\policy(s,a)\ 
		\forall a\in\Actions\ \forall s\in\States
\]
%%%

%%%
As previously discussed,
an optimal policy $\policy_*$ is any policy which maximizes the expected reward
received.
%
Furthermore, policies can be compared.
%
A policy $\policy$ is greater than another policy $\policy'$
if and only if the value of all states under policy $\policy$
are greater than or equal to the value of all states under $\policy'$
\[
\policy \ge \policy'\ 
	\iff\ v_\policy(s) \ge v_{\policy'}(s)\ \forall s\in\States
\]
%%%


\paragraph{Policy Evaluation}

%%%
% What does it mean to evaluate a policy
%%%

%%%
Policy evaluation is the iterative process of approximating the state-value
function $v_\policy$ for some policy $\policy$.
%
By repeatedly ... TODO ...
%%%

% Policy Evaluation algorithm

\begin{algorithm}
\caption{Policy Evaluation}
\label{alg:poleval}

\begin{algorithmic}[0]

	\Require $\policy$ (the policy), $\theta$ (some small number)

	\State Let $V[1...n]$ be an array of values for all states
	% where $n$ is the number of states
	$(n = |\States|)$

	\Repeat

		\State $\Delta \gets 0$

		\ForAll{$s \in \States$}

			\State $v \gets V[s]$

			\State $V[s] \gets
				\sum_a {\policy(a|s)}
				\sum_{s',r} {p(s',r|s,a)\left[r + \gamma V(s')\right]}$

			\State $\Delta \gets \max(\Delta,|v-V[s]|)$

		\EndFor

	\Until{$\Delta < \theta$}

	\State \Return $V \approx v_\policy$

\end{algorithmic}

\end{algorithm}


\paragraph{Policy Improvement and Iteration}

%%%
% 
%%%

	\subparagraph{Generalized Policy Iteration}

	%%%
	% A couple small notes about policy iteration in general
	%%%

\paragraph{Value Improvement}

%%%
% Overview of algorithm and how it's just sweeps of both on occasion
%%%

\paragraph{Monte Carlo Methods}

%%%
% Meaty crux of the thesis.
% An explanation of how policy improvement can be approximated using simulated
% results.
%%%

