
\subsubsection*{Rules of the Game}

%%%
In order to be able to understand the temporally dependent nature of the 
strategies, the rules and flow of a game of cribbage must be fully understood.
%
%While a more complete set of scoring rules can be found in Appendix
%~\ref{app:scoring_rules} and a complete set of tournament rules can be found
%at \cite{ACC_rules},
While a complete set of tournament rules can be found at~\cite{ACC_rules}
or~\cite{ACC_rulebook},
what follows is an overview complete enough such that a novice player,
with the assistance of the scoring rules found in Table~\ref{tab:app-score-rules},
could play a complete game.
%%%


\section{During Play Round}

\section{During Counting phase}


%%%
The zeroth step, taken once per game, is to determine which player will be the
dealer for the first round and who will be the pone.
%
In order to determine these roles,
each player cuts the deck in turn to get a single card:
the player with the lower-ranked card%
\footnote{Ace < 2 < 3 < 4 < 5 < 6 < 7 < 8 < 9 < 10 < Jack < Queen < King;
suits are irrelevant.}
is the dealer;
the other player is called the pone.
%\footnote{
In the case of a tie, this step is repeated until two unique cards are
cut from the deck.
%}
%
From there, the usual round structure begins and proceeds in the following
steps:
%
\begin{enumerate} %[noitemsep,topsep=0pt]
%
\item Each player is dealt 6 cards.
%
\item Each player selects 4 cards to keep for their own hand and 2 cards to
	discard, or toss, into a collective discard pile, called the crib.
%
\item The deck is cut at a random location by the pone
	and the top card from this cut is selected by the dealer and placed
	face-up on top of the deck.
	%
	If this cut card is a Jack of any suit,
	the dealer is awarded 2 points and pegs the points accordingly.
%
\item Starting with the pone, each player alternates playing a single card
	by placing them face-up on the table,
	keeping track of the total value\footnote{
		The value of all numbered cards is that number, aces have
		a value of 1, and all face cards have a value of 10.
	}
	of all cards played so far,
	until all cards have been played
	or neither player can play a card without exceeding a collective value of
	31.
	%
	If any of the situations or combination of cards mentioned in
	Table~\ref{tab:app-score-rules} is seen in the immediately preceding cards,
	the amount of points earned is immediately pegged on the board for the
	player who played the last card.
	%
	In cribbage terms, this is called \textit{the play}
	or, perhaps confusingly,
	in certain circles due to the rapid nature of the
	action, \textit{pegging}.
	%
	These terms will be used interchangeably throughout this paper,
	with a preference for \textit{pegging}.
	%
	This process is repeated until both players' cards have been exhausted.
%
\item After all cards have been played,
	the pone then counts his or her hand
	using the randomly cut card from Step 3 as a
	fifth card in the hand before pegging these points on the board.
%
\item The dealer proceeds to count his or her hand and peg the points
	in the same fashion, also considering the randomly cut card to be the
	fifth card in the hand.
%
\item The dealer then does the same for the crib.
%
\item The dealer and the pone swap roles and repeat from step 1.
%
\end{enumerate}
%
The game is immediately over when either player achieves a score
of 121.
%If, at any point, a player achieves a score greater than or equal to 121,
%that player is immediately declared the winner and the game is over.
%%%


