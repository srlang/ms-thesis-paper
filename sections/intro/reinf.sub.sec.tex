\subsection{Reinforcement Learning}
\label{sec:intro-reinf}

%%%
% Background of topics in reinforcement learning
%%%

%%%
% What is Reinforcement Learning:
% When is it used:
%%%

%%%
Reinforcement learning is the machine learning equivalent of learning from one's
failures,
rather than being coached to the correct answer.%
\footnote{
	Unless otherwise specified,
	all information presented in this section is referenced from
	Sutton and Barto~\cite{rl_book}.
}
%
In classical machine learning methodologies,
the agent discovers an optimum model for a problem by approximation methods
centered around minimizing a set loss function over a given set of data
while assuming a known model for the solution.
% ^^^ TODO: check if accurate enough to say
%
In reinforcement learning,
however,
the agent finds the optimal solution to the problem by repeatedly taking an
action in an environment and gaining a reward or punishment for each
action taken.
%
It is the same principle used in teaching a pet or animal to do a trick:
offer a full or partial reward for successful completion of the trick or
for progress in the correct direction.
%
As a comparison to teaching a human how to add,
the strategy used in classical machine learning would be what is used in
classrooms today:
teach the method of adding digits and handling carry-over,
giving some guidance and sample problems to ensure the technique is solidly
replicable.
%
Meanwhile, teaching a human how to add by reinforcement learning would
mean merely quizzing the subject by asking him to answer an addition problem
while giving a vague hint as to how right or wrong they were.
%
After enough rounds of this,
the student will eventually figure out his own method for adding two numbers
with the same accuracy as an established method.
%%%

%%%
While this may seem like a silly example where classical methods would clearly
be the superior method,
where reinforcement learning comes into its own is in situations in which no
known answer exists for a problem.
%
Take for instance the problem of learning chess.
%
In humans
basic strategies for how to handle certain situations can be taught,
but these are all from one's own or others' prior experience,
and while they may bolster knowledge on heretofore unknown situations,
they cannot possibly cover all possible chessboard layouts.
%
The best way to learn, in humans and computers, is by doing.
%
Barring savants such as Bobby Fischer,
humans will learn how to play chess better by playing games against varying
opponents.
%
After a game has been played,
the player can see what worked and failed during the game to cause the win or
loss and extrapolate what to do if a similar situation occurs.
%
Classical teaching methodologies would not be highly applicable in this
situation because there is no single correct strategy in chess that can
be taught.
%%%

%%%
% Loss functions in 'classical' machine learning
%	and how they are unusable in reinforcement learning scenarios
%	What is a reinf learning scenario?
%		when is it used/favored
%%%

\subsubsection*{Agent, Rewards, and Environment}

%%%
% Explanation of basic background topics in reinforcement learning
%	what is the environment
%	what is reward defined as
%	what is a policy,goal,etc.
%%%

%%%
The three most important, constantly interacting components in a reinforcement
learning scenario are
the agent, the environment, and the rewards.
%
The agent is the actor that makes decisions and learns the task at hand.
%
The agent must learn to navigate the environment in order to maximize its
rewards,
much like a mouse navigating a maze to retrieve the cheese at the end.
%%%

\paragraph*{The Environment}

%%%
In reinforcement learning scenarios,
the agent interacts with and navigates what is known as the environment.
%
The environment is a set of states in which an agent can find itself.
%
What exactly constitutes the environment is problem-specific.
%
An individual state can be any situation in which the agent finds itself
and can be in either discrete or continuous space.
%
For instance, in chess, a discrete state would be a specific board arrangement.
%
In golf, an example of a state in continuous space would be 
the location of the ball along the course of play
and the current wind velocity.
%
An action is an interaction the agent can make with the environment to alter
its current state.
%
In the example of chess,
an action is discrete and would be to move a piece X to position Y,
e.g.\  moving the bishop to g4 (\textit{Bg4}).
%
In golf, the action is continuous and may be
which club to use in which direction and with how much power.
%%%

\paragraph*{Rewards and Goals}

%%%
Merely being able to navigate an environment does not satisfy the
requirement for learning unless a given task is being completed.
%
This desired task can be called the goal of the agent.
%
For games scenarios,
this is frequently simply the notion of winning.
%%%

%%%
A reward is a feedback event that encourages or affirms progress
towards the goal
and 
can be thought of as a way of enticing the agent to accomplish the task.
%
As with a dog learning to jump through a hoop,
the reward of a treat is given after the dog has successfully jumped
through the hoop,
or perhaps a partial treat for first walking through a stationary hoop
or other similar subtask.
%
Rewards can also be negative
and thought of as punishments for not completing the task,
or doing so in an undesired manner.
%
The goal of the agent is to maximize its cumulative received reward
in the future.
%%%

%%%
Expressed mathematically, $R_t$ is the reward at a given time $t$.
%
A return, $G_t$, is the total reward expected to be received by the agent
in the future:
\[
	G_t = \sum_{k=t+1}^{T} R_k
\]
where $T$ is the final time step.
%
This return formula can also incorporate a discounting factor $\gamma$ to
encourage actions conducive to reaching the terminal state in a speedy fashion:
\begin{align*}
	G_t &= \sum_{k=t+1}^{T} \gamma^{k-t-1} R_k \\
		&= R_{t+1} + \gamma R_{t+2} + \gamma^2 R_{t+3} + \ldots \\
		&= R_{t+1} + \gamma G_{t+1}
\end{align*}
%
A rational agent will take an action to maximize its expected return.
%%%

%%%
Consider the scenario of trying to escape a maze.
%
The reward for the agent may be +1 when it successfully exits the maze
and 0 at all other times.
%
Without a discounting factor in the returns,
there would be no incentive to exit the maze in a timely fashion
as the return remains the same no matter how long the agent takes.
%
To handle this without a discount factor,
the rewards would need to be negative for all non-exit spaces
so that the return is worse the longer the agent remains in the maze.
%%%

%%%%
%In the context of cribbage,
%a reasonable reward formulation,
%and the one used in this thesis,
%would be the final point spread observed by the agent after playing the game.
%%
%Therefore,
%the agent's goal at any given state is to win
%by as many points as possible to increase the reward obtained,
%as quickly as possible to reduce the decay observed.
%%%%

\paragraph*{Policies}

%%%
A policy is a mapping of actions to states.
%
A policy $\policy$ describes a set of probabilities $P(a|s)$
of taking action $a \in \Actions$
when in state $s \in \States$
where $\Actions$ is the set of all possible actions
and $\States$ is the set of all states in the environment.
%
An optimal policy $\policy_*$,
of which there may be several,
is any policy which achieves a maximum expected reward over the course of
taking its actions.
%%%



\subsubsection*{Learning an Optimal Policy}

%%%
% A walk-through of policy evaluation, policy improvement, and value iteration
%%%

%%%
%For the sake of discussing how an optimal policy can be learned,
%it is useful from hereon to limit the scope of discussion to the case
%of a discrete set of states and actions.
%
Although applicable to both discrete and continuous state representations,
as long as it can be represented as a Markov decision process,
it is useful for the sake of illustration to limit the scope of discussion to
discrete representations.
%
Heretofore, %unless otherwise stated,
all discussion will assume a discrete representation
as the domain of cribbage falls into this category
and it streamlines notation.
%%%

\paragraph*{Metrics}

%%%
% Discuss what it means to be optimal/ pi_i > pi_j
% Explanation of value of a state: v_{pi}(s)
% Quality of an action q_{pi}(s,a)
%%%

%%%
A state can have a worth or \textit{value} associated with it,
for a given policy $\policy$.
%
The value of a state $s \in \States$ under policy $\policy$ is denoted
$v_\policy(s)$
and
is defined as the expected total return by following policy $\policy$ from
state $s$:
\begin{align*}
	v_\policy(s) &= \Expectation_\policy\left[G_t \middle| S_t = s\right] \\
		&= \Expectation_\policy \left[
				\sum_{k=0}^{\infty} \gamma^k R_{t+k+1} \middle| S_t = s
			\right]
		\\
		&= \sum_{a} {
				\policy(a|s)
				\sum_{s',\,r} {
					P(s',r|s,a) \left[
						r + \gamma v_{\policy}(s')
					\right]
				}
			}
\end{align*}
where $t$ is the arbitrary time within the episode
when the agent is in state $s$,
to maintain compatibility with previous definitions regarding rewards sequences.
%
The optimum value of a state $v_*(s)$ is defined as:
\[ v_*(s) = \max_\policy v_\policy(s) \]
%%%

%%%
Similarly,
an action can have its own worth or \textit{quality} assigned to it under a
specific policy.
%
The quality of an action $a\in\Actions$ at state $s\in\States$ under policy
$\policy$ is defined as:
\begin{align*}
	q_\policy(s,a)
		&= \Expectation_\policy\left[ G_t \middle| S_t = s, A_t = a \right]
		\\
		&= \Expectation_\policy\left[
				\sum_{k=0}^{\infty}\gamma^k R_{t+k+1} \middle| S_t = s, A_t = a
			\right]
\end{align*}
and its optimum $q_*(s,a)$ is consequently defined as:
\[
	q_*(s,a) = \max_{\policy}q_\policy(s,a)
\]
%%%

%%%
As previously discussed,
an optimal policy $\policy_*$ is any policy which maximizes the expected reward
received.
%
If there are optimal and suboptimal policies,
then it follows that there are ways in which policies can be compared.
%
A policy $\policy$ is greater than or equal to another policy $\policy'$
if and only if the value of all states under policy $\policy$
are greater than or equal to the value of all states under $\policy'$
\[
\policy \ge \policy'\ 
	\text{iff}\ 
	v_\policy(s) \ge v_{\policy'}(s)\ \forall s\in\States
\]
%%%


\paragraph*{Policy Evaluation}

%%%
% What does it mean to evaluate a policy
%%%

%%%
Policy evaluation is the iterative process of approximating the state-value
function $v_\policy$ for some policy $\policy$.
%
When the entire state space is observable,
since $v_\policy(s)$ can be expressed recursively,
the value of a state can be repeatedly updated
to its cumulative expected reward when taking each possible action,
converging to the true state-value function.
%
However,
as this convergence only occurs at the limit,
the calculation process can be stopped when its magnitude of change
indicates that it is sufficiently accurate,
as shown in Algorithm~\ref{alg:poleval}.
%
Due to its iterative nature,
however,
this can be a slow process
as each state can affect others in a cascading fashion.
%%%

% Policy Evaluation algorithm

\begin{algorithm}
\caption{Policy Evaluation}
\label{alg:poleval}

\begin{algorithmic}[0]

	\Require $\policy$ (the policy), $\theta$ (some small number)

	\State Let $V[1...n]$ be an array of values for all states
	% where $n$ is the number of states
	$(n = |\States|)$

	\Repeat

		\State $\Delta \gets 0$

		\ForAll{$s \in \States$}

			\State $v \gets V[s]$

			\State $V[s] \gets
				\sum_a {\policy(a|s)}
				\sum_{s',r} {p(s',r|s,a)\left[r + \gamma V(s')\right]}$

			\State $\Delta \gets \max(\Delta,|v-V[s]|)$

		\EndFor

	\Until{$\Delta < \theta$}

	\State \Return $V \approx v_\policy$

\end{algorithmic}

\end{algorithm}


\paragraph*{Policy Improvement and Iteration}

% Policy iteration algorithm

\begin{algorithm}
\caption{Policy Iteration}
\label{alg:politer}

\begin{algorithmic}[1]

\Require something
\Require Somethihg else



\end{algorithmic}

\end{algorithm}


%%%
After the value of a policy can be accurately estimated,
it is then possible to improve on that policy.
% 
This process is as simple as greedily choosing the action at each state such
that the expected resulting reward is maximized.
%
Intuitively,
this is an optimal policy,
given the current knowledge of the environment,
since each action taken will always make the most rational decision
in whichever state it finds itself in.
%
The idea of policy iteration is formed
from this simple greedy policy improvement mechanism.
%
Alternating steps of policy evaluation and policy improvement are repeated,
converging
until a stable policy is reached that is within some precision bound of optimal.
%
This process is given in Algorithm~\ref{alg:politer}
and can be visualized as:
\[
	\policy_0 \Earrow v_{\policy_0} \Iarrow
	\policy_1 \Earrow v_{\policy_1} \Iarrow
	\policy_2 \Earrow \ldots \Iarrow
	\policy_* \Earrow v_{\policy_*}
\]
where $\Earrow$ shows policy evaluation and $\Iarrow$ shows policy improvement.
%
Although Algorithm~\ref{alg:politer}'s conditional check
in Line 5 of Step 3,
written as is,
can allow cycling between multiple optimal policies,
the algorithm is illustrative
and care can be taken in implementation to remove this possibility.
%%%

	%\subparagraph*{Generalized Policy Iteration}

	%%%
	% A couple small notes about policy iteration in general
	%%%

\paragraph*{Value Iteration}

%%%
% Overview of algorithm and how it's just sweeps of both on occasion
As mentioned,
the iterative nature of policy evaluation leads to a slow convergence rate
for the process of policy iteration.
%
However,
full convergence does not need to occur for an optimal policy to be computed.
%
Instead,
a very truncated form of policy evaluation can be done to improve the speed
of evaluation while still converging to an optimal policy.
%
A special case of only performing one pass of policy evaluation and improvement
during policy iteration is called value iteration
and is shown in Algorithm~\ref{alg:valueiter}. 
%
This algorithm still converges to optimal policy,
just with less accurate steps made in improvement.
%
Rather than make a slow, calculated step in the absolute best direction,
value iteration takes a quick step in a generally good direction,
always improving its situation and
allowing later improvements in judgment to compensate for the potentially poor
steps previously taken.
%%%

% Value iteration algorithm

\begin{algorithm}
\caption{Value Iteration}
\label{alg:valueiter}

\begin{algorithmic}[0]

	\Require $V$ initialized arbitrarily
		(e.g. $V(s) = 0 \ \forall s \in \States$), %\\
		$\theta$ (some small number)

	\Repeat

		\State $\Delta \gets 0$

		\ForAll {$s \in \States$}

			\State $v \gets V(s)$

			\State $V(s) \gets
				\max_a \sum_{s',r} {
					p(s',r|s,a)\left[ r + \gamma V(s') \right]
				}$

			\State $\Delta \gets \max(\Delta, |v-V(s)|)$

		\EndFor

	\Until $\Delta < \theta$

	\State \Return Deterministic policy $\policy \approx \policy_*$,
		such that \\
		$\policy(s) = \argmax_a \sum_{s',r} {
					p(s',r|s,a)\left[ r + \gamma V(s') \right]
				}$

\end{algorithmic}

\end{algorithm}


\paragraph*{Monte Carlo Methods}

%%%
% Meaty crux of the thesis.
% An explanation of how policy improvement can be approximated using simulated
% results.
%%%

%%%
Policy iteration requires full knowledge of the environment to be computed.
%
However,
this is often not practical in real-world applications.
%
Monte Carlo methods are a set of ways in which
state-value functions, action-value functions, or policies can be estimated through experience without
complete knowledge of the environment.
%
These experiences can come from episodes in the real environment
or through simulated encounters.
%
By averaging observed returns from experienced episodes,
the state-value function,
and thus policy,
can be learned.
%%%

%%%
The key problem in learning through experience is how to balance
exploitation of the current value function or policy
to increase immediate returns
with exploration of the environment to discover potentially better returns.
%
An agent which explores too much
may not have very accurate knowledge of each action's eventual outcome
and often sacrifices better returns in the attempt.
%
Comparatively,
an agent which explores too little
may not discover that a better outcome is possible
than those which it already knows.
%
There are two mechanisms to combat this issue
and ensure the necessary exploration of the environment
while also exploiting current knowledge to maximize rewards.
%
The first,
using exploring starts,
starts each episode randomly in the environment's state-space.
%
An algorithm for this sort of Monte Carlo method is given in
Algorithm~\ref{alg:mces}.
%
This mechanism forces the agent to explore a different region
of the state space than it may likely reach on its own.
%
The other,
using an $\epsilon$-greedy policy,
follows the policy less strictly,
taking an action at uniform random with chance $\epsilon$.
%
Taking an action at random puts the agent in situations
just outside its normal operating area.
%
Whereas random starts can be thought of as global exploration,
random actions would be a local exploration step,
widening the knowledge area around previously known states.
%
Naturally,
both of these methods can be combined to further ensure adequate exploration.
%%%

% Monte Carlo with Exploring Starts

\begin{algorithm}
\caption{Monte Carlo with Exploring Starts}
\label{alg:mces}

Initialize for all $s \in \States, a \in \Actions(s)$:
\begin{itemize}[noitemsep]
	\item $Q(s,a) \gets$ arbitrary
	\item $\policy(s) \gets$ arbitrary
	\item $\textit{Returns}(s,a) \gets$ empty list
\end{itemize}


Repeat forever:
\begin{itemize}[noitemsep]
	\item Choose $S_0 \in \States$ and $A_0 \in \Actions(S_0)$
		such that all pairs have probability $> 0$
	\item Generate an episode starting from $S_0,A_0$, following $\policy$
	\item For each pair $s,a$ appearing in the episode:
		\begin{enumerate}[noitemsep]
		\item $G \gets$ return following the first occurrence of $s,a$
		\item Append $G$ to $\textit{Returns}(s,a)$
		\item $Q(s,a) \gets \text{average}(\textit{Returns}(s,a))$
		\end{enumerate}
	\item For each $s$ in the episode:
		\begin{itemize}[noitemsep]
			\item $\policy(s) \gets \argmax_a Q(s,a)$
		\end{itemize}
\end{itemize}

\end{algorithm}


