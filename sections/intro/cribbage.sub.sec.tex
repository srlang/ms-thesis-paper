
%%%
% An introductory overview of the game of cribbage.
% Complete in its coverage so that someone can start playing, but won't know
%	the exact scoring at all given times.
%%%

\subsection{Cribbage}

%%%
Cribbage is a multi-phase card game, typically played between two opposing
players.
%
While variants exist for three or more players, this paper will focus on the
two-player variant.
%
The game presents an interesting research area because of its unique scoring
methodology:
each hand is counted in two slightly different ways within each round and the
first player to reach a score of 121 points or more is declared the winner.
%
Players will usually keep track of their points by \textit{pegging} them on a 
characteristic board.
%
Because of its atypical win condition, different strategies hold differing
levels of importance throughout the game.
%%%


\subsubsection*{Rules of the Game}

%%%
In order to be able to understand the crucial nature of the temporally dependent
strategies, the rules and flow of a game of cribbage must be fully understood.
%
%While a more complete set of scoring rules can be found in Appendix
%~\ref{app:scoring_rules} and a complete set of tournament rules can be found
%at \cite{ACC_rules},
While a complete set of tournament rules can be found at~\cite{ACC_rules}
or~\cite{ACC_rulebook},
what follows is an overview complete enough such that a novice player,
following the scoring rules found in Appendix~\ref{app:scoring_rules},
could play a complete game.
%%%

%%%
The zeroth step, taken once per game, is to determine which player will be the
dealer for the first round and who will be the pone.
%
%The player who is not the dealer is called the pone.
%
In order to determine these roles,
each player cuts the deck once to get a single card:
the player with the lower-valued card%
\footnote{Ace < 2 < 3 < 4 < 5 < 6 < 7 < 8 < 10 < Jack < Queen < King}
is the dealer;
the other player is called the pone.
%\footnote{
In the case of a tie, this step is repeated until two unique cards are
cut from the deck.
%}
%
From there, the usual round structure begins and proceeds in the following
steps:
%
\begin{enumerate}
%
\item Each player is dealt 6 cards.
%
\item Each player selects 4 cards to keep for their own hand and 2 cards to
	discard, or toss, into what is called the crib.
%
\item A random card is cut from the remaining cards of the deck and placed
	face-up on top of the deck.
	%
	If this cut card is a Jack of any suit,
	the dealer is awarded 2 points and pegs the points accordingly.
%
\item Starting with the pone, each player alternates playing a single card,
	keeping track of the total value\footnotemark of all card played so far,
	until all cards have been played,
	or neither player can play a card without exceeding a collective value of
	31.
	\footnotetext{
		The value of all numbered cards is that number, aces have
		a value of 1, and all face cards have a value of 10.
	}
	%
	In the latter case, the player last to play a card will be awarded points
	before the count is reset and play continues.
	%
	If any of the combination of cards mentioned in
	Appendix~\ref{app:scoring_rules} is seen in the immediately preceding cards,
	the amount of points earned is immediately pegged on the board for the
	appropriate player.
	%
	In cribbage terms, this is called \textit{the play}
	or perhaps confusingly,
	in certain circles due to the rapid nature of the
	action, \textit{pegging}.
	%
	These terms will be used interchangeably throughout this paper,
	with a preference for \textit{pegging}.
%
\item After all cards have been played,
	the pone then counts his or her hand using the randomly cut card as a
	fifth card in hand before pegging these points on the board.
	%5\textsuperscript{th}
%
\item The dealer then proceeds to count his or her hand and peg the points
	in the same fashion, also considering the randomly cut card to be the
	fifth card in the hand.
	%5\textsuperscript{th}
%
\item The dealer then does the same for the crib.
%
\item The dealer and the pone swap roles and repeat from step 1.
%
\end{enumerate}
%
If, at any point, a player achieves a score greater than or equal to 121,
that player is immediately declared the winner and the game is over.
%%%




%akkari

%%%
The win condition for this game can occur at any moment of the game,
even beyond either player's control (note Step 3).
%
Because of this,
it is crucial to play according to different strategies during
different times of the game,
where time can be defined by what score the player has,
combined with what score their opponent has
and which player is the dealer for that round.
%
Typically, during early and middle-game play,
the pone will attempt to maximize their own hand,
while avoiding giving too much opportunity for the dealer to score points from
the crib.
%
% TODO: check rules for when However is acceptable as start word of sentence
However, in later play, this may no longer be a concern.
%
For example, should the player be the pone and their score is 116
and the dealer has 117 points,
due to the counting precedence,
the player needs not concern themselves with what points the dealer will obtain
through the crib, if their own hand has at least 5 points guaranteed,
since the pone will count first and win.
%
This only works, however, if the pone does not allow the dealer to score 4
points from \textit{the play}.
%
As can be seen, the player must balance multiple competing factors with varying
emphasis over the course of the game.
%%%

%%%
% TODO: visual example for the crowd
%	Good choice:
%	
%%%

