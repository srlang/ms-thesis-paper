
%%%
% An introductory overview of the game of cribbage.
% Complete in its coverage so that someone can start playing, but won't know
%	the exact scoring at all given times.
%%%

\subsection{Cribbage}

%%%
Cribbage is a multi-phase card game, typically played between two opposing
players.
%
While variants exist for three or more players, this paper will focus on the
two-player variant.
%
The game presents an interesting research area because of its unique scoring
methodology:
each hand is counted in two slightly different ways within each round and the
first player to reach a score of 121 points or more is declared the winner.
%
Players will usually keep track of their points by moving pegs along on a 
characteristic board
in a process called \textit{pegging}.
%
Because of its atypical win condition, different strategies hold differing
levels of importance throughout the game.
%%%


\subsubsection{Rules of the Game}

%%%
In order to be able to understand the crucial nature of the temporally dependent
strategies, the rules and flow of a game of cribbage must be fully understood.
%
While a more complete set of scoring rules can be found in Appendix
~\ref{app:scoring_rules} and a complete set of tournament rules can be found
at \cite{ACC_rules},
what follows is an overview complete enough such that a novice player 
%%%


%akkari

%%%
The win condition for this game can occur at any moment of the game,
even beyond either player's control (note Step 3).
%
Because of this,
it is crucial to play according to different strategies during
different times of the game,
where time can be defined by what score the player has,
combined with what score their opponent has
and which player is the dealer for that round.
%
Typically, during early and middle-game play,
the pone will attempt to maximize their own hand,
while avoiding giving too much opportunity for the dealer to score points from
the crib.
%
However, in later play, this may no longer be a concern.
%
For example, should the player be the pone and their score is 116
and the dealer has 117 points,
due to the counting precedence,
the player needs not concern themselves with what points the dealer will obtain
through the crib, if their own hand has at least 5 points guaranteed,
since the pone will count first and win.
%
This only works, however, if the pone does not allow the dealer to score 4
points from \textit{the play}.
%
As can be seen, the player must balance multiple competing factors with varying
emphasis over the course of the game.
%%%

