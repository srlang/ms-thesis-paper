\subsubsection*{Learning an Optimal Policy}

%%%
% A walk-through of policy evaluation, policy improvement, and value iteration
%%%

%%%
%For the sake of discussing how an optimal policy can be learned,
%it is useful from hereon to limit the scope of discussion to the case
%of a discrete set of states and actions.
%
Although applicable to both discrete and continuous state representations,
it is useful for the sake of illustration to limit the scope of discussion to
discrete representations.
%
Heretofore, unless otherwise stated, all discussion will assume a discrete
representation.
% TODO: ^^^ check that this is needed
%%%

% TODO: need a value function explanation (background knowledge, maybe?)

\paragraph*{Metrics}

%%%
% Discuss what it means to be optimal/ pi_i > pi_j
% Explanation of value of a state: v_{pi}(s)
% Quality of an action q_{pi}(s,a)
%%%

%%%
A state can have a \textit{value} associated with it given a policy $\policy$.
%
The value of a state $s \in \States$ under policy $\policy$ is denoted
$v_\policy(s)$
which can signal the ``worth'' of the given state
and is defined as the expected total reward by following policy $\policy$ from
state $s$:
\begin{align*}
	v_\policy(s) &= \Expectation_\policy\left[G_t | S_t = s\right] \\
		&= \Expectation_\policy \left[
				\sum_{k=0}^{\infty} \gamma^k R_{t+k+1} | S_t = s
			\right]
\end{align*}
%
The optimum value of a state $v_*(s)$ is defined as:
\[ v_*(s) = \max_\policy v_\policy(s)\ \forall s \in \States \]
%%%

%%%
Similarly,
an action can have its own ``worth'' or \textit{quality} assigned to it under a
specific policy.
%
The quality of an action $a\in\Actions$ at state $s\in\States$ under policy
$\policy$ is defined as:
\begin{align*}
	q_\policy(s,a)
		&= \Expectation_\policy\left[ G_t | S_t = s, A_t = a \right]
		\\
		&= \Expectation_\policy\left[
				\sum_{k=0}^{\infty}\gamma^k R_{t+k+1} | S_t = s, A_t = a
			\right]
\end{align*}
and its optimum $q_*(s,a)$ is accordingly defined as:
\[
	q_*(s,a) = \max_{\policy}q_\policy(s,a)\ 
		\forall a\in\Actions\ \forall s\in\States
\]
%%%

%%%
As previously discussed,
an optimal policy $\policy_*$ is any policy which maximizes the expected reward
received.
%
Furthermore, policies can be compared.
%
A policy $\policy$ is greater than another policy $\policy'$
if and only if the value of all states under policy $\policy$
are greater than or equal to the value of all states under $\policy'$
\[
\policy \ge \policy'\ 
	\text{iff}\ 
	v_\policy(s) \ge v_{\policy'}(s)\ \forall s\in\States
\]
%%%


\paragraph*{Policy Evaluation}

%%%
% What does it mean to evaluate a policy
%%%

%%%
Policy evaluation is the iterative process of approximating the state-value
function $v_\policy$ for some policy $\policy$.
%
By repeatedly ... TODO ...
%%%

% Policy Evaluation algorithm

\begin{algorithm}
\caption{Policy Evaluation}
\label{alg:poleval}

\begin{algorithmic}[0]

	\Require $\policy$ (the policy), $\theta$ (some small number)

	\State Let $V[1...n]$ be an array of values for all states
	% where $n$ is the number of states
	$(n = |\States|)$

	\Repeat

		\State $\Delta \gets 0$

		\ForAll{$s \in \States$}

			\State $v \gets V[s]$

			\State $V[s] \gets
				\sum_a {\policy(a|s)}
				\sum_{s',r} {p(s',r|s,a)\left[r + \gamma V(s')\right]}$

			\State $\Delta \gets \max(\Delta,|v-V[s]|)$

		\EndFor

	\Until{$\Delta < \theta$}

	\State \Return $V \approx v_\policy$

\end{algorithmic}

\end{algorithm}


\paragraph*{Policy Iteration and Improvement}

% Policy iteration algorithm

\begin{algorithm}
\caption{Policy Iteration}
\label{alg:politer}

\begin{enumerate}

	\item Initialization \\
		$V(s) \in \Reals$
		and
		$\policy(s) \in \Actions(s)$ arbitrarily for all $s$

	\item Policy Evaluation (see Algorithm~\ref{alg:poleval}).


	\item Policy Improvement
		\begin{algorithmic}[0]
			\State $\textit{policy-true} \gets \textit{true}$

			\ForAll{$s \in \States$}

				\State $\textit{old-policy} \gets \policy(s)$

				\State $\policy(s) \gets
					\argmax_{a} \sum_{s',r} {
						p(s',r|s,a)\left[r+\gamma V(s')\right]
						}$

				\If {$\textit{old-action} \neq \policy(s)$}
					\State $\textit{policy-stable} \gets \textit{true}$
				\EndIf

			\EndFor
		\end{algorithmic}

\end{enumerate}

\end{algorithm}


%%%
% 
%%%

	\subparagraph*{Generalized Policy Iteration}

	%%%
	% A couple small notes about policy iteration in general
	%%%

\paragraph*{Value Improvement}

%%%
% Overview of algorithm and how it's just sweeps of both on occasion
%%%

\paragraph*{Monte Carlo Methods}

%%%
% Meaty crux of the thesis.
% An explanation of how policy improvement can be approximated using simulated
% results.
%%%
