
\subsection{Strategies}
\label{sec:dm-methods-strategies}

%%%
% Information which strategies were created and how they basically operated
%	side note about the need to create a different script
%%%

%%%
%During the simulation of games,
A few basic behavioral strategies were coded for the agent to learn
over the course of multiple simulated games.
%
These represented most of the basic factors which a player may consider when
making their decision as to which cards to keep
as well as some that are not quite as useful.
%
These included:
\begin{itemize}
\item \handmaxmin:
	%
	The hand(s) which has the highest minimum possible score for the kept cards
	will be more highly desired.
	%
	This is equivalent to choosing the hand with the largest guaranteed points.
	%
\item \handmaxavg:
	%
	The hand(s) with the maximum average points over all possible cut cards will
	be the most highly desired.
	%
	This strategy is useful for trying to maximize the expected score of one's
	own hand.
	%
\item \handmaxmed:
	%
	The hand(s) with the maximum median points possible to score will be the most
	highly desired.
	%
	%Similar to \texttt{hand\_max\_avg}, this strategy is useful when trying to
	%maximize one's hand's expected score.
	%
\item \handmaxposs:
	%
	The hand(s) with the maximum possible score will be the most highly desired.
	%
	This strategy can be thought of as a Hail Mary choice for the player trying
	to score as many points as possible, not taking into account its
	potentially low likelihood to become reality.
	%
\item \cribminavg:
	%
	The hand(s) whose tossed cards led to cribs with the lowest average amount
	of points for the dealer is
	the most highly desired.
	%
	This strategy would be a very defensive strategy typically used by the pone
	to avoid giving points to the dealer.
	%
\item \peggingmaxavggained:
	%
	The hand(s) with the maximum average points gained through pegging is the most
	highly desired.
	%
	This strategy would be useful for end-game play in a tight game.
	%
	For instance, if both players are close to winning, the dealer may
	choose to forego placing points into their hand and instead try to
		``peg out''
	since it is unlikely that they will get a chance to count their hand at all.
	%
\item \peggingmaxmedgained:
	%
	The hand(s) with the maximum median points scored during play will be the most
	highly desired.
	%
\item \peggingminavggiven:
	%
	The hand(s) with the minimum average points scored by the opposing player
	will be the most highly desired.
	%
	This is a very defensive strategy also useful at the end of the game to
	prevent the opposing player from pegging out.
	%
\end{itemize}
%
The above definitions refer to a hand's desirability.
%
This can be thought of as an internal ranking of how likely a given strategy
would be willing to choose a particular combination of cards.
%
This desirability score is then scaled to lie in the range $[0, 1]$ with
$1$ representing the best possibilities.
%
This scaling allows for possibilities which are ``almost as good'' to not be
ignored in later weighting stages.
%%%


%\subsubsubsection*{Database Population}
%%%
Because of the massive amount of possible combinations%
\textemdash all of which were unique due to the cards' position affecting the
scoring outcome\textemdash%
the evaluation of some of these possibilities in which the crib was involved
in real-time took a handful of seconds to evaluate during development.
%
This was deemed much too slow as,
over the course of tens or hundreds of thousands of simulated games,
this delay would very quickly accumulate to performance-affecting delays.
%
As a result, an alternative strategy of pre-computing the required knowledge and
storing the values of interest into a database was implemented instead.
%%%

%%%
There are ${52 \choose 6}$ possible combinations of cards which can be dealt to
either player.
%
Of these possibly dealt hands, there are then ${6 \choose 4} = 15$ possible
combinations of cards that can be kept,
with the remainder ``tossed'' to the crib.
%
For each of these $15$ possible combinations,
there are a further $46$ remaining possible cards
that must be considered as possible cut cards for the kept hand.
%
Additionally, for the thrown set of cards,
there are ${46 \choose 2} = 1035$ possible combinations of cards which can be
thrown by the opposing player into the crib as well as the
$44$ remaining possible cut cards.
%
These cut cards must be considered separately in such a manner of
evaluating ${46 \choose 2} \times 44 = 45540$ possibilities rather than
simply looking at each of ${46 \choose 3} = 15180$ possibilities
since the presence of the card in the crib or as the cut card
can affect the score.\footnote{
	For example, as per the right-jack (his nobs) scoring rule,
	a hand of 5\clubs\ 5\hearts\ 5\spades\ J\diamonds\ 
	with a cut card of 5\diamonds\ is a perfect hand
	with a score of 29,
	but moving those cards around so that the jack is no longer in the hand
	and is instead the cut
	(i.e.\  hand of 5\clubs\ 5\hearts\ 5\spades\ 5\diamonds\
	with a cut of J\diamonds)
	yields a score of 28.
}
%
Just as crucially, there are small differences in scoring the crib as opposed to
scoring the player's own hand that mean that previous evaluations' results are
not reusable.
%
For instance, the rule for gaining points from a flush are more lenient for
one's own hand than for the crib:
the crib's flush must be a five-card flush containing the cut card
whereas the player's own hand needs only be a four-card flush of their own
hand with bonus points gained from the crib also matching.
%
This means that, altogether, there are 
\[
	{52 \choose 6}
	{6 \choose 4}
	\left(
		%\left(46\right) +
		%\left(
			{46 \choose 2} \times 44
		%\right)
	\right)
	\approx
	1.391 \times {10}^{13}
	% Google answer: 1.3920952e13
	% Right google answer: 1.3906905e13
	%684790
\]
possible combinations of cards that need to be evaluated in total to fully 
understand the statistics
of a cribbage game for every set of cards that can possibly be dealt.
%%%

%%%
Although the majority of this project was coded in Python for its ease of use
and speed of development,
due to the performance-crucial, basic mathematical operations involved,
the overheads of using a higher-level language was deemed too critical and
this particular tool was developed in C instead.
%
To put the performance gains into perspective,
at its fastest, most parallelly processed state,
the Python database populator was estimated to take approximately 4 months to
simply list and evaluate all possible scores on a development machine,
disregarding the file I/O operations required to write those results to the
database.
%
The same functionality, with the addition of file I/O,
in the C program would take a relatively mere 14 days
on that same machine.
%
Furthermore, access to a high-performance server allowed for further
parallelization which cut the final run time down to approximately 5 and a half
days.
%%%

%%%
Careful consideration and preparation needed to be taken
for the retrieval of this information, however.
%
The trillions of combinations could not be quickly searched by card value as
doing so would require searching the entirety of the database on each lookup,
decreasing the performance so much as to be worse than simply
enumerating the combinations on-demand.
%
A rather simple solution to this problem was to search by index
instead of by card comparison.
%
These indices could not simply be stored in the memory of the running program
because the sheer size required to store all of the indices at all times
would rival that of the database,
and its population would take considerable enough time.
%
Thus, a quick and reliable method for creating and recreating these keys
needed to be used.
%
As the cards were represented internally as an integer between $0$ and $51$
(inclusive)
and there were only 6 cards for indexing,
the concatenation of the cards' digits in (keep,toss) order
would create a number with at most twelve digits,
with an absolute maximum value of 484950514647,
well within the range of numbers addressable by a 64-bit integer.\footnote{
	Thanks to implementation,
	the order of cards was guaranteed to be sorted within each tuple,
	so each combination of cards was tracked and not permutation.
}
%
This index could be created by a very simple process of 6 multiplications and
5 additions
while still being guaranteeably unique.
%%%

%%%
While the scores for combinations of chosen and tossed cards could be evaluated 
before any games had actually been played,
the hands' usefulness during the pegging portion of the round needed to be
evaluated in semi real-time.
%
A single pegging agent was programmed with a simple one-ahead greedy heuristic:
the card that gained the most points when played next was selected.
%
Ties were
broken by choosing the highest-valued card that reached that score
in order to potentially outmaneuver the opponent from reaching a count of 31.
%
This agent was then played against a copy of itself with randomly allocated
cards and the results of that round were recorded for each agent to provide an
initial knowledge base.
%
These results would then be queried by the agent during the choose phase of the
game and contributed to at the end of the pegging phase as training progressed.
%%%

