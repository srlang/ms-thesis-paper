
%%%
% TikZ diagram here of matrix multiplication and where each vector/matrix comes
% from
%%%

\begin{figure}[h]
\centering

%\begin{tikzpicture}
%	\matrix (mod) {4 & 5}
%	\matrix (p) [right=of mod] { 1 & 5 & 6 \\ 3 & 5 & 5}
%	\matrix (S) [right=of p] {1 & 5 \\ 3 & 5}
\begin{tikzpicture}[cell/.style={rectangle,draw=black},
space/.style={minimum height=1.5em,matrix of nodes,row sep=-\pgflinewidth,column sep=-\pgflinewidth,column 1/.style={font=\ttfamily}},text depth=0.5ex,text height=2ex,nodes in empty cells]

\matrix (mod) [space, column 1/.style={font=\ttfamily},column 2/.style={nodes={cell,minimum width=2em}}]
{
0   & 6 \\   1   & 3 \\   2   & 9 \\   };

\matrix (p) [right=of mod, space, column 2/.style={minimum width=3em,nodes={cell,minimum width=3.5em}},column 3/.style={nodes={cell,minimum width=2em}}]
{
0   &a  & 6 \\   1   &   & 3 \\   2   &c  & 9 \\   };

\matrix [right=of p, space, column 2/.style={minimum width=3em,nodes={cell,minimum width=3.5em}},column 3/.style={nodes={cell,minimum width=2em}}]
{
0   &a  & 6 \\   1   &b  & 3 \\   2   &c  & 9 \\   };
\end{tikzpicture}
	\caption{A visual depiction of the weighting operation.}
	\label{fig:matmul-proc}
\end{figure}
