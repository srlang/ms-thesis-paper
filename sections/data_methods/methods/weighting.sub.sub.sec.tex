
%%%
% How the different strategies were weighted
%%%

\subsubsection{Weighting}

%%%
% Introductory stuff
%%%

%%%
% More mathematically expressed
%%%

%%%
For the purposes of this project,
how well certain combinations of cards are played with different strategies
is not explored.
%
Instead, only the player's position in score-space affects the decision as to
which strategies to play by.
%
Put another way, the agent does not care what cards it is dealt as much as where
it is located on the board.
%
Each possible score-space location can be thought of as a discrete coordinate
defined by the parameters
$\textit{PlayerScore} \in [0, 120]$,
$\textit{OpponentScore} \in [0, 120]$,
and
$\textit{Dealer?} \in \{0,1\}$.
%(\textit{MyScore:$[0,120]$, OpponentScore:$[0,120]$, Dealer?:$\{0,1\}$}).
%
At each score-space location is a vector $w_{p,o,d} = [w_1,w_2,\ldots,w_m]$
where $m$ is the number of all possible strategies to be considered.
% TODO: ^: how do we want to notate this? w_i? explain f_i is a float?
At the beginning of each round, each of $m$ strategies is evaluated for all
$n = {6 \choose 4}$ possible combinations of cards kept to produce an
$m \times n$ matrix $S$
where $S_{i,j}$ is the desirability of the $i^{\textit{th}}$ keep/toss
combination according to the $j^{\textit{th}}$ strategy
further constrained by
$0 \le S_{i,j} \le 1\ \forall i,j$.
%$\sum_{i=1}^{m} S_{i,j} = 1\ \forall j$
%and $\sum_{j=1}^{n} S_{i,j} = 1\ \forall i$.
%
A value vector $P$ of length $n$ representing the total perceived value of a
possible keep combination
can then be computed by
$P = w S$
wherein $\argmax_{x}{P_x}$ can be thought of to be the most desired combination
of cards and $\argmin_{x}{P_x}$ as the least.
%
These collective desirability metrics can be later used to determine which
combination of cards to choose and which to toss.
%%%


