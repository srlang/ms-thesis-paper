
%%%
% How the different strategies were weighted
%%%

\subsubsection{Weighting}

%%%
% Introductory stuff
%%%

%%%
% More mathematically expressed
%%%

%%%
For the purposes of this project,
how well certain combinations of cards are played with different strategies
is not explored.
%
Instead, only the player's position in score-space affects the decision as to
which strategies to play by.
%
Each possible score-space coordinate can be thought of as a coordinate defined
by (\textit{MyScore:$[0,120]$, OpponentScore:$[0,120]$, Dealer?:$\{0,1\}$}).
%
At each score-space location is a vector $w_{M,O,D} = [f_1, f_2,\ldots,f_m]$
where $m$ is the number of all possible strategies to be considered.
%
At the beginning of each round, each of $m$ strategies is evaluated for all
$n = {6 \choose 4}$ possible combinations of cards kept to produce an
$m \times n$ matrix $S$
where $S_{i,j}$ is the SOMETHING TODO
and $\sum_{i=1}^{m} S_{i,j} = 1\ \forall j$
and $\sum_{j=1}^{n} S_{i,j} = 1\ \forall i$.
%
A value vector $P$ of length $n$ representing the total perceived value of a
possible keep combination
can then be computed by
$P = w S$.
%
%where $n$ is the number of possible combinations of cards that are kept and
%thrown into the crib,
%i.e. ${6 \choose 4} = 24$.
%%%


