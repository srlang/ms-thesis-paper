
\subsubsection*{Implementation Decisions}
\label{sec:disc-future-implementation}

%%%
Due to the desire to get a training framework constructed
and running as quickly as possible,
runtime speed was deemed less important than development speed.
%
As a result,
Python became the language of choice.
%
While this is a great multipurpose language,
the overhead of using a higher-level, interpreted language
proved to be a mistake
not only directly because of its slower speed involved with the
lower-level calculations and manipulations needed for cribbage applications,
but also because of resulting decisions that needed to be made to accommodate
this reduced speed.
%%%


\paragraph*{Database Dependency}

%%%
Because Python was the language of choice,
the performance of making a decision for a single hand was sluggish:
on a development machine,
the decision for a single hand would take approximately ten seconds.
%
As previously mentioned in Section~\ref{sec:dm},
the decision made from that point was to create a database such that all
calculations were done ahead of time
and the statistics desired would be available upon request.
%
This succeeded in its desired goal,
bringing the time required for a single decision to approximately
one twentieth of a second.
%
At this point the project was able to play two games in about a second's time
on the same development machine.
%%%

%%%
Despite the speed improvements in a development environment,
the entire training process had become I/O-bound rather
than CPU bound.
%
This became a problem when  an attempt was made to train multiple
agents simultaneously.
%
Because of concurrency issues which would have
taken too long to fix or work around,
no more than one process could successfully access the SQLite database with any
consistency.
%
This would mean that only two agents could be trained using a single database
at any given time.
%
Furthermore,
since this database file was around twenty gigabytes in size and
there was only limited local storage space available
on each of the compute nodes in the Computer Science Department's Ukko2 high
performance cluster,
jobs had to be constructed carefully and spread across nodes to avoid
accidentally disrupting other users' processes.
%
This was mitigated slightly by using the Physics Department's Kale cluster
which allowed for significantly more manageable locally mounted drive space,
albeit in a slightly slower hard disk with a RAID0 configuration.
%%%

%%%
Were this project to be repeated,
it would be recommended to use the Python code from this attempt as an outline
for a rewrite in a lower-level language.
%
Using a language such as C++,
which could be optimized more for run-time efficiency,
would likely increase speed of computation enough to remove the necessity
of a statistics database entirely.
%
While the data gathered from previously played pegging rounds would still need
to be stored and retrieved between training rounds,
the total size of this data would likely not exceed a few gigabytes and would fit
easily in the program's RAM space.
%
This is especially true if only
minimums, maximums, and averages
are considered,
as their calculation does not require all values previously seen to be stored,
meaning that far less storage space is needed.
%
Also,
like weight checkpoints are in the current system,
these stored statistics could be exported
for transfer between rounds.
%%%

%%%
With these changes made,
the training program would be CPU-bound,
which can be compensated for by parallelization.
%
In contrast,
the I/O-bound training program used in this thesis used only half the total
processing power of a single CPU core
as the bottleneck was the database file on a hard drive.
%%%

