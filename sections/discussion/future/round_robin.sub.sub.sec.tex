
\subsubsection*{Multi-agent Play}
\label{sec:disc-future-rr}

%%%
% Discussion of how round-robin play could improve play and learning
%%%

%%%
In all of the tournament training sessions and other experiments,
two agents were paired against each other for a million games.
%
%While the opponent for each agent changed due to the learning process,
%this process was gradual and each agent only had to ever adapt to a single
%
It is rare for humans to only play a single opponent
for an extended period of time.
%
Only in tournaments,
or in absence of other available opponents,
would a human player play more than a couple of games
against the same opponent.
%
Although the reinforcement learning algorithm requires far more games to
achieve similar progress in learning,
there is no need to limit the agent to a single opponent.
%%%

%%%
As an alternative,
the agent could be pitted against multiple agents,
each with their own unique set of
weights.
%
This would force the agent to learn how to overcome a variety of opposing
strategies,
strengthening the overall performance of the agent.
%
One potential solution to this is to have a pool of agents
from which the agent will randomly select an opponent.
%
Similarly,
two agents could be chosen from a pool at random and pitted against each
other,
in a round-robin fashion.
%%%

%%%
After the promising results of the experiments using different starting points
for the agents,
the idea of learning how to overcome multiple, widely varying agents
rather than a single co-adapting agent
seems to be the most promising starting point for continued research into this
topic.
%
While learning from self-play was idealistic
and even demonstrated possible by AlphaGo Zero~\cite{deepmind_alphago_zero},
it was perhaps naïve as
the need for some priming was a crucial step in earlier AlphaGo
work~\cite{deepmind_alphago} and in TD-Gammon~\cite{tdgammon}.
%
As an alternative to pure self-play,
and in the absence of recorded games,
an agent representing each of the most potent basic strategies
(e.g.\  \handmaxavg, \handmaxmin, and \handmaxposs)
could similarly be played against and used to train the learning agent
to basic levels before self-play is used for further improvement.
%%%

