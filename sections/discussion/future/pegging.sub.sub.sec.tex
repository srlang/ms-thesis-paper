% Pegging future work

\subsubsection{Pegging}

%%%
% Discussion of what can be done for learning pegging better
% Shortcomings:
%	1. only a basic heuristic
%		- can train a player for this alone
%	2. estimate performance on incomplete/small amount of data
%		- pre-populated a bit, but ~50-100k, not millions
%			(definitely missing some combos when first round started)
%%%

%%%
% Basic intro paragraph
%%%

%%%
One area of the project that can be improved was the pegging system.
%
Because the focus of the thesis was on whether or not a collection of
strategies could be learned in combination,
an ideally playing agent was not the outcome.
%
A key reason for this was the lack of time dedicated to and nonchalant nature
towards learning how to actually play the crucial pegging phase.
%%%

\paragraph{Playing Strategy}

%%%
% Discussion of Failures of basic heuristic
%	- very basic strategy only looks one immediate step ahead
%	- no context awareness as to whether to play offensively/defensively
%		(ironic)
%%%

%%%
The first shortcoming of the pegging portion of the agent was the simplicity
of its playing strategy.
%
In order to focus on the overall playing strategy and to have a consistent
knowledge of how a set of cards have performed in the past,
the agent was only capable of following a very simple immediately greedy 
heuristic:
of the cards left which are legal to play,
play the one with the highest immediate return.
%
From the cribbage player's perspective this has the potential issue of
opening oneself up to the possibility of allowing the opponent an opportunity
to score more than the agent itself just gained,
resulting in a net loss.
%
As has been demonstrated by the rest of this thesis,
the idea that one strategy can be applied at all times in the game of cribbage
is laughable.
%so the single heuristic used is 
%%%

%%%
In the future,
a partial agent could be trained in just the pegging phase alone
using reinforcement learning.
%
In a similar way to this thesis,
multiple basic strategies such as offensive or defensive play
could be trained by rewarding each behavior and their combinations
could in turn be learned through gameplay simulations.
%
Alternatively,
a more ground-up approach can be taken to train an unbiased agent by simply
having each player learn from trial and error with certain card combinations.
%
This has the disadvantage of taking more time to train to effectiveness.
%
In either case,
a more optimized pegging system can be made that would play better.
%%%

\paragraph{Performance Evaluation}

%%%
Even though a better pegging system can be made,
key to this thesis project was the idea that performance can be predicted
or at least anticipated with some amount of certainty.
%
In this project,
the performance of the pegging agent was tracked by card combination,
recording the points scored and yielded during each occurrence.
%
These records were used to anticipate how each combination of cards would
play against an opponent.
%%%

%%%
Before the first round of training,
however,
there was relatively little pre-population of these records done:
roughly one hundred thousand randomly dealt cards were played against each
other.
%
As a result of the small sample size,
performance estimates were likely to be inaccurate.
%
Furthermore,
the small sample size did not demonstrably cover all of the possible hand
combinations.
%
Because of this,
data would be missing for multiple combinations of cards
for the decision process
throughout the first round.
%
This would mean that the card combinations' true desirability would be
misrepresented during the calculation,
allowing for the possibility of a philosophically unfair
punishment or reward for what would amount to a guess by the
pegging-based strategies.
%%%

