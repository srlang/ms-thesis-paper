% Pegging future work

\subsubsection*{Pegging}
\label{sec:disc-future-pegging}

%%%
% Discussion of what can be done for learning pegging better
% Shortcomings:
%	1. only a basic heuristic
%		- can train a player for this alone
%	2. estimate performance on incomplete/small amount of data
%		- pre-populated a bit, but ~50-100k, not millions
%			(definitely missing some combos when first round started)
%%%

%%%
% Basic intro paragraph*
%%%

%%%
One area of the project that can be improved was the pegging system.
%
Because the focus of the thesis was on whether or not a collection of
strategies could be learned in combination when choosing which cards to keep,
an ideally playing agent was not the outcome.
%
A key reason for this was the lack of time dedicated to and nonchalant nature
towards learning how to actually play the crucial pegging phase.
%%%

\paragraph*{Playing Strategy}

%%%
% Discussion of Failures of basic heuristic
%	- very basic strategy only looks one immediate step ahead
%	- no context awareness as to whether to play offensively/defensively
%		(ironic)
%%%

%%%
The first shortcoming of the pegging portion of the agent was the simplicity
of its playing strategy.
%
In order to focus on the overall playing strategy and to have a consistent
knowledge of how a set of cards have performed in the past,
the agent was only capable of following a very simple immediately greedy 
heuristic:
of the cards available which are legal to play,
play the one with the highest immediate return.
%
This has the potential flaw of
opening oneself up to the possibility of allowing the opponent an opportunity
to score more than the agent itself just gained,
resulting in a net loss.
%
As has been demonstrated by the rest of this thesis,
the idea that one strategy can be applied at all times in the game of cribbage
is laughable.
%so the single heuristic used is 
%%%

%%%
In future applications,
a partial agent could be trained in just the pegging phase
using reinforcement learning.
%
In a similar way to this thesis,
multiple basic strategies such as offensive or defensive play
could be trained by rewarding each behavior and their combinations
could in turn be learned through gameplay simulations.
%
Alternatively,
a more ground-up approach can be taken to train an unbiased agent by simply
having each player learn from trial and error with certain card combinations.
%
%This has the disadvantage of taking more time to train to effectiveness.
%%%

%%%
On the other hand,
as cribbage is an imperfect-information game,
a parallel can be drawn to the game of poker.
%
Since pegging is the phase of cribbage
which most depends on predicting and countering the opponent's strategy,
recent successes by Dr.\ Tuomas Sandholm and Noam Brown
in heads-up no-limits Texas hold'em may be applicable to
this phase.
%
Their AI, \textit{Libratus},
defeated four top human players
in a 120,000-hand tournament
\cite{sandholm_poker}.
%
Furthermore, while \textit{Libratus} utilized massive computational resources,
Sandholm and Brown have also developed \textit{Modicum},
a less powerful, but less computationally expensive, poker-playing agent
which utilizes only hardware that can be found in moderately powerful desktop computers
\cite{sandholm_poker_improved}.
%
Although not as well-performing as \textit{Libratus},
\textit{Modicum} was still capable of beating other previous top poker agents
using fewer resources by orders of magnitude,
allowing for its use in research by those without access to high-performance machines.
%
In any case,
a more suitable pegging system can be made,
improving play.
%%%

\paragraph*{Performance Evaluation}

%%%
Even though a better pegging system can be made,
key to this thesis was the idea that performance can at least be
anticipated with some amount of certainty.
%
In this project,
the performance of the pegging agent was tracked by card combination,
recording the points scored and yielded during each occurrence.
%
These records were used to anticipate how each combination of cards would
play against an opponent.
%%%

%%%
Before the first round of training,
however,
there was relatively little pre-population of these records done:
roughly one hundred thousand randomly dealt hands were played against each
other.
%
As a result of the small sample size,
performance estimates were likely to be inaccurate.
%
Furthermore,
the small sample size did not cover all of the possible hand
combinations.
%
Because of this,
data would be missing for multiple combinations of cards
for the decision process
throughout the first round.
%
This means that the card combinations' true desirability would be
misrepresented during the calculation,
allowing for the possibility of an unfair
punishment or reward for what would amount to a guess by the
pegging-based strategies.
%
It also means that the information received by the agent during the weighting
operation would not be static, with respect to the cards given,
as other strategies were and thus could not be as
reliably trusted for accuracy.
%
While
this was counteracted by using records from a first round training session in 
all subsequent training sessions,
the concern remains whether this was enough
and how much the variability in the data provided by the pegging strategies
affected learning.
%%%

